\chapter{Aim 1}

\section{Abstract}
Hypohydration (HYPO) is believed to impair cognitive function with higher level processing (e.g., executive control, working memory) disproportionately affected. Purpose: To complete a systematic literature review and meta-analysis examining if DEH impairs cognitive tasks requiring executive functions. Methods: Pubmed, Web of Science, EBSCO, Scopus, PsychInfo, and Sport Discus were searched using keywords: *hydration, water loss, weight loss, hypovol*, sweat loss, cognition, and other specific cognitive function terms.  Thirteen studies were identified, providing data on 200 human subjects with DEH levels ranging from 1.2 to 4.7 \% body mass loss. Methods to induce DEH, control conditions, outcome variables, and executive function tasks varied between studies. Effect sizes (ES) were calculated using standardized mean differences and a random effects meta-analysis was utilized.  Results: Five of 13 studies reported DEH impaired (p < 0.05) executive functions (either reduced accuracy or increased reaction time). The overall ES of DEH impairment on executive functions was small (ES = - 0.31), but non-significant (p = 0.14; 95\% CI [- 0.74, 0.12]) and exhibited a high degree of heterogeneity among effects (I${^2}$ = 82.5 \%). The ES did not differ when isolating reaction time and accuracy. Conclusion: In contrast to narrative reviews, a meta-analysis indicates DEH does not significantly impair higher level cognitive processing when assessed by tests of executive functions. This is likely a result of inconsistencies across studies.

\section{Introduction}
Despite a number of studies attempting to understand the impact of HYPO on cognition function, the evidence remains inconsistent \cite{grandjean_dehydration_2007,masento_effects_2014,lieberman_hydration_2007,wilson_impaired_2003}. While initial studies suggested mental arithmetic, executive function, and information processing were impaired by HYPO \cite{gopinathan_role_1988, sharma_influence_1986}, although current studies have not uniformly supported these findings \cite{adam_hydration_2008, ely_hypohydration_2013}. The direct reason behind the conflicting evidence is unclear; however, some have suggested the variability within study design (e.g., magnitude of HYPO, method of inducing HYPO) and the parameters of cognition assessed are large contributors \cite{lieberman_hydration_2007, lieberman_methods_2012,masento_effects_2014}. Despite these inferences, there has not been a quantitative analysis of the overall effects of HYPO on cognitive performance or further analysis on moderators which would speak to issues with cognitive performance data. 

Although the literature is conflicting, extreme levels of HYPO consistently have elicited impairments of cognitive performance \cite{king_brief_1878, adolf_physiology_1947}. Soldiers under adverse environments have impaired ability to navigate, complete military operations, and, if severe enough, result in the presentation of confusion and delirium \cite{king_brief_1878,adolf_physiology_1947}. More recently, soldiers undergoing 5\% body mass loss during a 72h training exercise had impairments vigilance, reaction time, attention, memory, and reasoning by 2-4 fold compared to rest \cite{lieberman_severe_2005}. However, these large degradations in performance are typically observed with confounding variables which are known to alter cognitive performance such as sleep deprivation \cite{krause_sleep-deprived_2017} and hypoglycemia \cite{strachan_acute_2001}. Furthermore, these military scenarios also include high levels of stress which can impair cognitive performance \cite{opstad_performance_1978}, and therefore the relative contribution of HYPO to these large deficits is unclear. 

Originally, it was believed that cognitive performance decrements are observed when body mass losses exceed 2\% \cite{gopinathan_role_1988,sharma_influence_1986,sharma_differential_1983}, an similar magnitude required to elicit physiological compensation (e.g., increased plasma osmolality) \cite{cheuvront_dehydration:_2014}. However, a direct relationship between HYPO and cognitive performance impairments has not been established. Increased HYPO magnitude and cortisol levels previously have been associated with decreases in working memory performance \cite{mcmorris_heat_2006}, however the inclusion of heat stress in this study may have: i) independently increased cortisol concentration or ii) impair cognitive performance \cite{hancock_effects_2003}. While others have also suggested a relationship between cognitive performance decrements with increasing body mass losses \cite{gopinathan_role_1988}, the relationship between cognitive performance and cortisol levels during HYPO potentially can provide unique insight into physiological mechanisms leading to impaired cognitive performance. 

Therefore, our purpose was to utilize quantitative analytical techniques (i.e., meta-analysis) to determine whether the previous literature indicates if HYPO impairs cognitive performance. By completing this systematic review and meta-analysis, we aim to to provide clarity within the largely confusing literature to determine the overall effect of HYPO on cognitive performance along with some potential mediator analysis. We hypothesize that, similar to previous narrative reviews, HYPO will induce a small negative effect on cognitive performance. Secondly, we hypothesize that HYPO magnitude will have a significant relationship with the magnitude of cognitive performance impairment. Lastly, we hypothesize there will be a significant difference in those studies hit a body mass loss ${>}$2\% compared to those eliciting under 2\% body mass loss.  
 
\section{Methods}
A systematic review was conducted on the research literature for the effects of HYPO and cognitive performance. Cognitive performance was operationally defined as any measurable outcome resulting from completion of a cognitive function task (e.g., reaction time, accuracy). The literature search was last completed on September 2017. Searches were conducted in the following databases accessible through the Georgia Institute of Technology library: PubMed, Medline, Psych Info, Sport Discuss, Web of Science, SCOPUS, ProQuest Theses and Dissertations, which collectively returned 8305 results (6591 without duplicates). References from relevant review articles were also examined \cite{lieberman_hydration_2007,masento_effects_2014,benton_small_2015} for articles not uncovered previously. Search terms to obtain studies relevant to this meta-analysis consisted of: ``dehydration”, ``hypohydration”, ``water loss”, ``hypovolemia”, ``hypovolemic”, ``sweat loss”, ``fluid restriction”, ``cognition”, ``cognitive function”, ``cognitive performance”, ``intelligence”, ``mental”, ``expertise”, ``recall”, ``executive function”, ``response time”, ``reaction time”, ``memory”, ``perception”, ``vigilance”, ``pattern recognition”, ``processing”, ``recognition”, ``sensory”, ``decision making”, ``attention”, and ``mood”. 

\subsubsection{Inclusion Criteria}
Studies meeting the following inclusion criteria were considered for review: i) the study was conducted on healthy (i.e., no clinical conditions) humans, ii) the study contained at least two time points (either in one or two sessions) or separate groups of subjects in which cognitive testing was completed following HYPO in one trial (or group) and under control conditions in the other, iii) changes in hydration status were reported using previously established biomarkers \cite{cheuvront_physiologic_2013}, and iv) some form of behavioral change (e.g., reaction time) was reported. Studies were excluded if the dehydration intervention occurred over a prolonged duration (${>}$ 48 h).     

\subsubsection{Selection of Studies}
A total of 6591 relevant publications were originally identified through the database searches. Of those, 6512 were initially excluded on the basis of the title and/or review of the abstract. Therefore, 79 publications underwent full-text examination for suitability in the current analysis. Of those 79, 48 were excluded, leaving a total of 31 articles to be included within the meta-analysis, all but one of which were published. A summary of decisions made is presented in Figure \ref{fig:prisma}.

%%% PRISMA DIAGRAM
\begin{figure}
  	\centering
  	\includegraphics[width=12cm]{figures/"prisma_ma".pdf}
  	\caption{Prisma diagram depicting the systematic review protocol in determining the population of studies for inclusion within the meta-analysis.}
  	\label{fig:prisma}

\end{figure}

\subsubsection{Data extraction}
Given the relatively small number of suitable studies, all parameters of cognition were included in the analysis. Each task was classified as one core parameter of cognition according to classifications defined in previous studies \cite{chang_effects_2012}. In circumstances where the cognitive assessment could be considered a multi-faceted assessment of cognition, task categorization was determined in accordance to author description. Continuous moderator variables were coded according to their actual value (i.e., subject age) whereas categorical moderator variables were coded with dummy variables corresponding to different levels of the variable. One primary moderator considered was the degree of HYPO. If not reported, BM measures were converted to a percent change score (\% Change = (BMpost - BMpre) / BMpost). Degree of HYPO was therefore examined as a moderator variable to assess whether the extent of body water losses (and resulting physiological perturbation) was related to alterations in cognitive performance. 

Secondly, various methods to induce HYPO were coded into the category best matching methods employed from the following: exercise, heat exposure, exercise-heat stress (exercise + heat exposure), fluid restriction (${>}$ 12 h), or fluid restriction (${>}$ 12 h) with additional exercise/heat stress. The control condition within the study was also coded as being either rest (no exercise), exercise with fluid replacement, heat exposure with fluid replacement, or more than one control condition. If HYPO was induced via an exercise (or exercise-heat stress) protocol, potential positive effects on cognition, as has been found previously \cite{chang_effects_2012} was controlled for by coding the latency from end of exercise to cognitive testing of being either less than or greater than 15 min. If information was provided about subject fitness level (objective or subjective), this information was captured by placing subjects into a category of either sedentary, low fit / unfit, moderately fit, fit / highly fit. Additionally, basic subject characteristics were coded such as subject gender (percent male subjects) and age (y). Cognitive dependent variables were further grouped into either of two forms: reaction time or accuracy. For results not presented in these two forms, data was coerced to either of the two groupings (e.g., errors were considered a measure of accuracy, speed as reaction time).       

\subsubsection{Meta-Analysis}
The extracted cognitive performance data were converted to a standard format by calculating the standardized mean change score using the metafor package for R (v1.9-9, www.metafor-project.org), which will be called the effect size (ES). Standardized mean change scores are beneficial in studies with psychological studies which have varying measurement scales \cite{morris_combining_2002}. In studies where the correlational data are not known, r was estimated from the average of studies with known HYPO-control correlation \cite{wittbrodt_fluid_2015,watson_mild_2015}. For the effect size estimate, Hedge’s \textit{g} was employed to minimize the inherent bias of Cohen’s \textit{d} to overestimate the effect size when standardized mean differences are used with small samples sizes.   

Given the nature of most studies examining the effects of HYPO on cognitive performance, many dependent variables were presented as outcomes measurements. Multiple effect sizes are problematic for most conventional meta-analysis studies, as the dependent structure of results (e.g., decreased reaction time but increased accuracy) may confound and compromise validity of the meta-analysis results unless the covariance structure of the results is known \cite{scammacca_meta-analysis_2014}. Given that no study in the selected studies presented this data, other meta-analytic methods were required. Recently, a robust variance estimation meta-analytical technique was developed which accounts for the within-study dependence by estimating the covariance matrix \cite{hedges_robust_2010}. This technique was therefore utilized in the current study within the robumeta package in R (\textit{cran.r-project.org/web/packages/robumeta}) using the \texttt{robu()} function. Furthermore, this technique has been recently refined to include small-sample corrections (${<}$ 40 studies) \cite{tipton_small_2015}, which was used given the current selection of 26 studies. The small-sample corrections utilize a calculated degrees fo freedom using the Satterwaite approximation (df = 2 / cv${^2}$, where cv is the coefficient of variation) \cite{tipton_small_2015}. However, because degrees of freedom are calculated, any value ${\le}$ 4 represents an elevated Type I error risk (${>}$ 10\%) and results may not represent population values \cite{tipton_small_2015}. Rho (${\rho}$), the value accounting for dependencies of effect sizes was set at the recommended 0.8 \cite{tanner-smith_robust_2014}. A subsequent sensitivity analysis revealed that changes in ${\rho}$ (from 0 to 1) did not influence the overall effect size (maximum change = 0.0002) in the current study.

For all the analyses, a negative effect size represents that HYPO impaired performance on a given cognitive test whereas a positive effect size represents an improvement. The robust variance estimation technique outputs the standardized mean difference between groups. Therefore, a significant covariate indicates that the effect size estimate depends on the value of that covariate.

\section{Results}

\subsection{Study Characteristics}
The final sample consistent of 31 studies (m), all but one was published. In total, there were 253 effect sizes (k; per study: min = 1, mean = 8.16, median = 6, max = 36) representing a total of 388 subjects. The characteristics of each study is presented in Appendix \ref{App:deh_cog}. 

\subsection{Primary Analyses}
Figure \ref{fig:forest_plot} presents effect sizes of all studies examining the effect of HYPO on cognitive-motor performance. Overall (m = 31, k = 253), HYPO elicited a significant reduction in cognitive performance (g${^+}$ = -0.28, t(29.3) = -4.4, p = 0.0001, 95\% CI: [-0.41, -0.15]). There was substantial heterogeneity within studies (I${^2}$ = 68.60), however the between-study heterogeneity was acceptable (${\tau}$${^2}$ = 0.18). This suggests that, although results within the study were disparate, the between-study results were relatively consistent. When examining only reaction time based outcomes (m = 23, k = 76), HYPO impaired cognitive-motor performance (g${^+}$ = -0.20, t(21.6) = -2.3, p = 0.03, 95\% CI: [-0.38, -0.02]) and studies appeared to exhibit similar within-study (I${^2}$ = 69.2) and between-study (${\tau}$${^2}$ = 0.17) heterogeneity. Accuracy based outcomes (m = 29, k = 177) were also significantly impaired by HYPO (g${^+}$ = -0.25, t(27.1) = -3.5, p = 0.002, 95\% CI: [-0.41, -0.10]) with similar within studies (I${^2}$ = 64.7) and between-study (${\tau}$${^2}$ = 0.15) heterogeneity.

\subsection{Moderator Analyses}
When controlling for the magnitude of HYPO (\% body mass loss), the significant overall cognitive performance impairment persisted (g${^+}$ = -0.28, t(27.93) = -4.3, p = 0.0002, 95\% CI: [-0.41, -0.14]), although there was no relationship between the magnitude of HYPO to level of cognitive performance impairment (slope = -0.03, t(5.2) = -0.70, p = 0.51). The significant overall main effect was also preserved when controlling for subject age (g${^+}$ = -0.29, t(26.3) = -4.2, p = 0.0003, 95\% CI: [-0.43, -0.15]), although there was no significant association between subject age and magnitude of impairment (slope = 0.013, t(5.7) = 0.20, p = 0.85), however our systematic review resulted in a small subject age range (range: 19.1 - 34.0 y). When examining methods of eliciting HYPO, exercise and exercise-heat stress were the most common HYPO methods utilized and most likely to induce cognitive-motor impairments (Table \ref{tbl:moderators}). Lastly, when descriptors/measurements of subject fitness level were presented, the included studies featured only recreationally or highly fit subjects. However, both recreationally fit and highly fit individuals had impaired cognitive-motor performance following HYPO (p ${<}$ 0.05; Table \ref{tbl:moderators}).  

%%% FOREST PLOT
\begin{figure}
	\centering
	\includegraphics[width=13cm]{figures/"forest_plot".pdf}
	\caption{Forest plot of overall studies examining the effect of hypohydration on cognitive-motor performance. There was an overall significant main effect (g${^+}$ = -0.28, p = 0.0001). Negative effect sizes (ES) indicate that hypohydration impaired cognitive-motor performance whereas positive ES indicates improvement. Box size indicates relative weight attributed to each effect size within the study.}
	\label{fig:forest_plot}
\end{figure}

%%% MODERATOR TABLE
\begin{table}
	\caption{Moderator Analysis on the relationship of categorical variables on cognitive performance. ${^a}$df ${<}$ 4, potential Type I error rate ${>}$ 0.10; g${^+}$: effect size, SE: standard error of the effect size estimate; df: degrees of freedom; t: t test statistic for the covariate; p: p value for the t test, m: number of studies, k: number of outcomes}
	\centering
	
	\begin{tabular}{lcccccc} 
		\hline
		\textbf{Variable} & \textbf{g${^+}$} & \textbf{SE} & \textbf{df} & \textbf{p} & \textbf{m} & 
		\textbf{k} \\
		\hline
		\textit{Method of Dehydration} &&&&&& \\
		Exercise & -0.29 & 0.12 & 6.0 & \textbf{0.05} & 15 & 46 \\
		Heat Stress & -0.22 & 0.14 & 4.0 & 0.19 & 5 & 31 \\
		Exercise-Heat Stress & -0.22 & 0.10 & 14.6 & \textbf{0.04} & 16 & 177 \\
		Fluid Restriction & -0.50 & 0.15 & 2.1${^a}$ & 0.07 & 4 & 9 \\
		Fluid Restriction + EHS	& -0.18 & 0.06 & 1.0${^a}$ & 0.19 & 2 & 7 \\
		&&&&&& \\
		\textit{Subject Fitness} &&&&&& \\
		Recreationally Fit & -0.28 & 0.11 & 11.5 & \textbf{0.02} & 13 & 126 \\
		Highly Fit &  -0.32 & 0.10 & 10.9 & \textbf{0.01} & 12 & 82 \\ 
		\hline	    
	\end{tabular}
	\label{tbl:moderators}
\end{table}

\subsection{Executive Function}
Figure \ref{fig:ex_func_ma} presents the study-specific and overall results for tasks of executive functions assessed following HYPO. Executive functions represented a subset of the overall dataset limited to tasks contained within the executive functions (e.g., executive control, working memory, information processing) \cite{shields_does_2015}. The final sample consisted of twenty one studies (m), yielding a total of 93 outcomes (k, range: 1 - 18 per study, mean = 2.36), with all data in being from published manuscripts. Overall, HYPO had a small, but significant, impairment on executive function (m = 21, k = 93, g${^+}$ =  -0.24, t(19.5) = -2.7, p = 0.01, 95\% CI [-0.43, -0.06]). There was large heterogeneity between-studies (I${^2}$ = 69.2) but acceptable within-study heterogeneity (${\tau}$${^2}$ = 0.20). There was a significant impairment in accuracy based outcomes within tests of executive functions (g${^+}$ = -0.29, t(18.3) = -2.8, 95\% CI: [-0.52,-0.07], p = 0.01) within similar within (I${^2}$ = 66.4) and between (${\tau}$${^2}$ = 0.18) study heterogeneity. However, when examining reaction-based outcomes during test of executive functions, HYPO did not impair cognitive-motor performance (g${^+}$ = -0.02, 95\% CI: [-0.34, 0.29], p = 0.87) with similar levels of within-study (I${^2}$ = 66.3) and between-study (${\tau}$${^2}$ = 0.16) heterogeneity.

When controlling for the magnitude of HYPO (mean = 2.3\% BM loss), the significant impairment in executive function was unaffected (g${^+}$ =  -0.19, p = 0.02), however, there was no significant relationship between level of magnitude and executive function impairment (slope = -0.06, 95\% CI: [-0.21, 0.09], p = 0.41). A similar pattern was also observed for subject age (g${^+}$ = -0.19, 95\% CI: [-0.35, -0.03], p = 0.02), with no relationship to cognitive-motor impairment (slope = 0.03, 95\% CI: [-0.14, 0.19], p = 0.73). When examining categorical moderators, no method to induce HYPO significantly impaired executive function performance (p ${>}$ 0.05, Table \ref{tbl:ex_func_moderators}). However, it does appear that moderately subjects had greater executive function performance impairments following HYPO (p ${<}$ 0.05) compared to highly fit subjects (Table \ref{tbl:ex_func_moderators}).

%%% EXECUTIVE FUNCTION FOREST PLOT
\begin{figure}
	\includegraphics[width=14cm]{figures/"ex_function_forest".pdf}
	\caption{Forest plot of overall studies examining the effects of hypohydration on executive functions. The overall effect was not significant (g${^+}$ = -0.24, p = 0.01). Negative effect sizes (ES) indicate that hypohydration impaired executive functioning whereas positive ES indicates improvement. Box size indicates relative weight attributed to each effect size within the study.}
	\label{fig:ex_func_ma}
\end{figure}

%%% EX FUNCTION MODERATOR TABLE
\begin{table}
	\caption{Moderator analysis for variables contained within studies examining executive functioning following hypohydration. ${^a}$df ${<}$ 4 (Type I error rate ${>}$ 0.10); g${^+}$: effect size, SE: standard error of the effect size estimate; df: degrees of freedom; t: t test statistic for the covariate; p: p value for the t test, m: number of studies, k: number of outcomes}
	\centering
	\begin{tabular}{lcccccc} 
		\hline
		\textbf{Variable} & \textbf{g${^+}$} & \textbf{SE} & \textbf{df} & \textbf{p} & \textbf{m} & 
		\textbf{k} \\
		\hline
		\textit{Method of Hypohydration} &&&&&& \\
		Exercise & -0.16 & 0.15 & 3.0${^a}$ & 0.37 & 4 & 8 \\
		Heat & -0.21 & 0.14 & 4.0 & 0.19 & 5 & 20 \\ 
		Exercise-Heat Stress & -0.23 & 0.13 & 10.8 & 0.11 & 12 & 61 \\
		&&&&&& \\
		\textit{Subject Fitness} &&&&&& \\
		Moderately Fit & -0.34 & 0.15 & 9.7 & \textbf{0.05} & 11 & 54 \\
		Highly Fit & -0.18 & 0.12 & 4.0 & 0.22 & 5 & 17 \\
		\hline		    
	\end{tabular}
	\label{tbl:ex_func_moderators}
\end{table}

\subsubsection{Motor Functioning}
Because this thesis specifically investigates the effects of HYPO on cognitive-motor performance, performance outcomes relative to the cognitive-motor system were examined to determine whether HYPO may impair cognitive-motor functioning. There were relatively few studies examining tasks under `motor coordination' (or similar), so for this analysis both the motor coordination studies and `occupation specific' studies were grouped together, as occupation specific tasks include a large motor component.

The final sample consisted of five studies (m; Table \ref{tbl:m_c_studies}), yielding a total of 14 outcomes (k, range: 1 - 6 per study, mean = 2.8). All data in the currents study were from published manuscripts. Overall, HYPO impaired cognitive-motor performance (m = 5, k = 14, g${^+}$ = -0.437, t(3.6) = -2.99, p = 0.046, 95\% CI [-0.861, -0.0133] (Figure \ref{fig:mc_forest}). There was large heterogeneity between-studies (I${^2}$ = 64.0) but acceptable within-study heterogeneity (${\tau}$${^2}$ = 0.13). Magnitude of HYPO was not significantly correlated to magnitude of cognitive-motor performance impairments (slope = 0.05, p = 0.80). Furthermore, when controlling for HYPO magnitude, the overall effects size decreased (g${^+}$ = -0.53, t(2.2) = -2.25); however, the added variability (95\% CI: [-1.459, 0.395]) resulted in this not being significant (p = 0.14). It should be added, however, that both the overall (df = 3.6) and subsequent HYPO magnitude (df = 1.4) analyses both were ${<}$4, which increases Type I error to ${>}$0.1. Therefore, more data is needed to help clarify whether or not HYPO impairs cognitive-motor performance.   

%%% MOTOR FUNC STUDIES
\begin{table}
	\caption{Studies utilizing aspect of cognitive-motor functioning or occupation specific task following hypohydration (HYPO). ND = No difference.}
	\centering
	\begin{tabular}{llll} 
		\hline
		\textbf{Study} & \textbf{\% BM Loss} & \textbf{Task} & \textbf{Self-Reported Findings} \\
		\hline
		Sharma et al. \cite{sharma_influence_1986} & 1-3 & Psychomotor Stylus Test & Impaired at ${\ge}$2\% BM loss \\
		Patel et al. \cite{patel_neuropsychological_2007} & 2.5 & Balance Error Scoring System & \textemdash \\
		Cian et al. \cite{cian_influence_2000} & 2.8 & Unstable Tracking Task & Increased Deviation (p ${<}$ 0.05) \\
		Wong et al. \cite{wong_effects_2014} & 1.4 - 2 & Psychomotor Reaction Time Test & Decreased Speed (p ${<}$ 0.05) \\
		Weber et al. \cite{weber_dehydration_2013} & 2.4, 4.8 & Concussion Test & Decreased Performance (p ${<}$ 0.05) \\
		\hline		    
	\end{tabular}
	\label{tbl:m_c_studies}
\end{table}


\begin{figure}
	\includegraphics[width=12cm]{figures/"mc_forest_plot".pdf}
	\caption{Forest plot of overall studies examining the effects of hypohydration on cognitive-motor performance. The overall effect was not significant (g${^+}$ = -0.437, p = 0.04). Negative effect sizes (ES) indicate that hypohydration impaired executive functioning whereas positive ES indicates improvement. Box size indicates relative weight attributed to each effect size within the study.}
	\label{fig:mc_forest}
\end{figure}

\subsubsection{Studies with ${>}$2\% BM Loss Induced via Exercise-Heat Stress}
Multiple reviews \cite{lieberman_hydration_2007,lieberman_methods_2012} have suggested the largely contradictory results in studies examining cognitive performance following HYPO is heavily influenced by the variety of methods to induce and magnitude of HYPO. Given that this thesis will elicit moderate BM loss via exercise-heat stress, performance outcomes from studies mirror this study design (2\% BM loss and exercise heat stress) were evaluated.

The final sample consisted of nine studies (m), yielding a total of 79 outcomes (k, range: 1 - 18 per study, mean = 8.8), with all data being from published manuscripts. The overall effect of HYPO on cognitive performance on this subset of studies (m = 9, k = 79) was not statistically significant, g${^+}$ =  -0.337, t(7.9) = -1.73, p = 0.122, 95\% CI [-0.787, 0.113] (Figure \ref{fig:ehs_forest_plot}). There was large heterogeneity between-studies (I${^2}$ = 77.4) but acceptable within-study heterogeneity (${\tau}$${^2}$ = 0.38). There was no significant relationship between HYPO magnitude (range: 2.0 - 4.65\% BM loss) and magnitude of cognitive performance (slope = 0.01, p = 0.96). When controlling for HYPO magnitude, the effect size was not significant (g${^+}$ =  -0.374, p = 0.55). Furthermore, even after controlling for all moderator variables, no significant effect of HYPO on executive functioning was observed (Table \ref{tbl:ehs_2_moderators}). 

%%% EHS & 2% MODERATORS
\begin{table}
	\caption{Moderator analysis for studies eliciting hypohydration via exercise-heat stress and achieving at least 2\% body mass loss. ${^a}$df ${<}$ 4 (Type I error rate ${>}$ 0.10); g${^+}$: effect size, SE: standard error of the effect size estimate; df: degrees of freedom; t: t test statistic for the covariate; p: p value for the t test, m: number of studies, k: number of outcomes}
	\centering
	\begin{tabular}{lcccccc} 
		\hline
		\textbf{Variable} & \textbf{g${^+}$} & \textbf{SE} & \textbf{df} & \textbf{p} & \textbf{m} & 
		\textbf{k} \\
		\hline
		\textit{Variable Type} &&&&&& \\
		Accuracy & -0.332 & -0.20 & 7.9 & 0.14 & 9 & 68 \\
		Reaction Time & -0.114 & 0.065 & 1.9${^a}$ & 0.227 & 3 & 11 \\
		\hline		    
	\end{tabular}
	\label{tbl:ehs_2_moderators}
\end{table}


%%% EHS and 2% FOREST PLOT
%\begin{figure}
%	\includegraphics[width=12cm]{figures/"ex_function_forest".pdf}
%	\caption{Forest plot of overall studies utilizing exercise-heat stress to induce hypohydration and eliciting ${\ge}$ 2\% body mass loss. The overall effect was not significant (g${^+}$ = -0.337, p = 0.112). Negative effect sizes (ES) indicate that hypohydration impaired executive functioning whereas positive ES indicates improvement. Box size indicates relative weight attributed to each effect size within the study.}
%	\label{fig:ehs_forest_plot}
%\end{figure}


\section{Discussion}
Dehydration may potentially impair cognitive-motor performance and potentially contribute to the increased workplace accidents and occupational risk \cite{kenefick_hydration_2007}. While various narrative reviews have suggested HYPO may impair cognitive performance, previous research has not unanimously supported this position \cite{cheuvront_dehydration:_2014}. To that extent, we instituted a quantitative analysis of the previous literature using a meta-analysis to understand the objective effect of HYPO on cognitive-motor performance. The main finding of this study is that HYPO had a small but significant negative effect on cognitive-motor performance, which appeared to be a result of accuracy-based outcomes versus reaction time-based outcomes.

The small, but significant, negative effect of HYPO on cognitive-motor performance aligns with some narrative reviews suggesting HYPO may only have a small effect on cognitive-motor performance, similar to other nutritional interventions \cite{masento_effects_2014}. However, one unexpected finding was the marked difference between reaction time and accuracy based outcomes. Accuracy-related outcomes (i.e., errors, percentage correct) were likely responsible for the overall cognitive-motor impairment following HYPO, with reaction time outcomes seemingly unaffected by body water losses. The likely explanation is the differences within what reaction time or accuracy represent to cognitive-motor functioning. Reaction time is largely influenced by the visual identification of a stimulus combined with a motor response. Multiple reports have observed no impairment to visual discrimination \cite{grego_influence_2005, van_den_heuvel_independent_2017} or maximal finger tapping frequency \cite{bandelow_effects_2010} following HYPO, suggesting resiliency within reaction time processes. Contrarily, accuracy based outcomes were impaired following HYPO, albeit to a small magnitude. Accuracy-based outcomes likely represent decisions faced within various occupational scenarios, and therefore may indicate judgment/decision making processes are impaired following HYPO. The physiological mechanism for this observed impairment are not fully understood, although some have demonstrated increased neural activation (i.e., inefficient processing) following HYPO \cite{kempton_dehydration_2011,watson_mild_2015}. 

To further probe the literature, subsets of the overall meta-analysis were isolated to quantitatively examine potential confounding variables hypothesized to influence the findings with regards to HYPO and cognitive-motor performance. Controlling for HYPO (expressed as a percent of body mass loss) did not alter the small, but significant overall impairment in cognitive-motor performance following HYPO, but no relationship was observed between magnitude of HYPO and cognitive-motor impairment. However, two factors may have influenced these findings: i) the small overall effect size which potentially limits the ability to find a graded relationship and ii) minimal studies which induced multiple magnitudes of HYPO. In one such study \cite{gopinathan_role_1988}, greater levels of HYPO induced greater cognitive-motor impairments across the domains of crystallized intelligence, executive function, and short-term memory. While these studies within-study designs offer control over various study confounding variables, not all studies have observed marked changes, with some observing no significant exacerbation of cognitive-motor impairment with greater levels of HYPO \cite{weber_dehydration_2013} or no differences at all \cite{baker_dehydration_2007}. The second moderator variable examined was whether the method to elicit HYPO impacted cognitive-motor responses. It appears that, when HYPO is induced via exercise-heat stress or exercise alone, cognitive-motor performance is significantly impaired. This agrees with ideas presented previously that methods to induce HYPO may contribute to the equivocal literature \cite{lieberman_methods_2012}, although how this manifests itself has not been previously understood. One interesting observation was the fluid restriction had a larger computed impairment on cognitive-motor performance (g${^+}$ = 0.50), although the Satterthwaite approximations \cite{tipton_small_2015} resulted in a degrees of freedom level inducing a Type I error rate ${>}$ 10\%, likely as a result from a paucity of available data (4 studies, 9 effects). Therefore, more research is needed to provide ample resolution of this potential impairment in cognitive-motor performance following fluid restriction protocols.    

The second important finding of this study was that executive functions (e.g., executive function, working memory) were impaired to a small but significant amount following HYPO. Various studies have suggested higher-order cognitive-motor components may be disproportionately impaired by HYPO \cite{tomporowski_effects_2007, lieberman_hydration_2007,grandjean_dehydration_2007}. The proposed mechanism as to why executive functions may be disproportionately impaired by HYPO has not been throughly presented. HYPO has previously increased neural resource requirements during an executive function test despite no impairment in performance in adolescents \cite{kempton_dehydration_2011}.  

Thirdly, we observed a moderate impairment of cognitive-motor and occupational task performance following HYPO. However, these results may not be robust, as the calculated df was ${<}$ 4 (3.4). Secondly, it should be noted that this meta-analysis did not include studies without data suited for inclusion in the meta-analysis, including studies finding an overall improvement \cite{bandelow_effects_2010} or no change \cite{hogervorst_cognitive_1996} in maximal button tapping speed following HYPO and another finding flight simulation impairments but without accurate BM changes \cite{lindseth_effects_2013}. The included studies utilizing many different facets of cognitive-motor performance such as emphasizing motor coordination (e.g., unstable track task, stylus task), stability, and motor-related reacting time task. Four of the five studies reported significantly cognitive-motor impairments following HYPO, with the lone study utilizing a more gross-motor task. This follows the trend of some \cite{savoie_effect_2015} but not all \cite{baker_progressive_2007} studies observing no impairment in gross-motor functioning following HYPO. 

Lastly, this meta-analysis observed no significant impairment of cognitive performance in studies employing exercise-heat stress to elicit ${\ge}$ 2\% BM loss. Although the magnitude of effect was similar to the overall effect (g${^+}$ = -0.337), the added bewteen-study variability prevented the effect from being significant. Interestingly, of the nine studies, four had a positive effect following HYPO, including two with 95\% confidence intervals being exclusively positive \cite{danci_voluntary_2009,bijlani_effect_1980}. Interestingly, the overall effect size had minimal variability when controlling for the magnitude of HYPO (g${^+}$ = -0.374), which potentially indicates that, once a magnitude of BM loss sufficient to up-regulate homeostatic mechanisms  \cite{sawka_american_2007,cheuvront_dehydration:_2014}, cognitive performance changes are similar. This conflicts with an early study observing initial cognitive impairments at 2\% BM loss and additive impairments at 4\% BM \cite{gopinathan_role_1988}.  

Because cortisol provides negative feedback from vasopressin release from the paraventricular nucleus \cite{andreoli_endocrine_2010}, elevated plasma concentrations indicate hypothalamus-pituitary-adrenal axis activation \cite{lieberman_severe_2005}, or magnitude of the thirst response. Therefore, it is possible HYPO-mediated cognitive impairments may not directly result from body water losses, but instead distractors such as thirst. Thirst is known to activate various brain areas such as the anterior cingulate \cite{saker_regional_2014} and other studies have suggested drinking small quantities of water increase memory and attention in children \cite{benton_effect_2009}. Furthermore, perceptual measures during HYPO....


 