\chapter{Aim 1: Does Hypohydration Impair Cognitive-Motor Performance: A Systematic Review and Meta Analysis}

\section{Abstract}
Hypohydration (HYPO) is believed to impair cognitive-motor performance, although not all studies have supported this position. PURPOSE: To complete a systematic literature review and meta-analysis examining if HYPO impairs cognitive-motor tasks. METHODS: A systematic review was completed to retrieve studies examining all facets of cognitive-motor performance following HYPO. Thirty-one studies were identified, providing data on 388 human subjects with HYPO levels ranging from 1.1 to 6.2\% body mass loss. Methods to induce HYPO, control conditions, outcome variables, and type of cognitive-motor tasks varied between studies. Effect sizes (ES) were calculated using standardized mean differences and a random effects meta-analysis was utilized. RESULTS: The overall ES of HYPO impairment on cognitive-motor performance was small (ES = -0.28), but significant (p = 0.0001; 95\% CI [-0.41, -0.152]) and exhibited a high degree of heterogeneity among effects (I${^2}$ = 68.6\%). Both accuracy (ES = -0.25) and reaction-time (ES = -0.20) outcomes were significantly affected (p $<$ 0.05). Subsequent analyses also determined executive functions (e.g., working memory, information processing, executive function) were impaired by HYPO (ES = -0.24, p = 0.01) as was cognitive-motor performance in studies eliciting ${\geq}$2\% body mass loss (ES = -0.31, p = 0.006). In all analyses, HYPO magnitude (\% body mass loss) was not significantly related to level of cognitive-motor impairment. CONCLUSION: This meta-analysis indicates HYPO significantly impairs cognitive-motor function. This impairment does not appear to be related to magnitude of HYPO, and therefore the observed impairments may be related to some other mechanism.


\section{Introduction}
Despite numerous studies investigating the impact of hypohydration (HYPO) on cognitive-motor performance, the evidence remains equivocal \cite{grandjean_dehydration_2007,masento_effects_2014,lieberman_hydration_2007,wilson_impaired_2003}. While initial studies suggested mental arithmetic, executive function, and information processing were impaired by HYPO \cite{gopinathan_role_1988, sharma_influence_1986}, recent studies have not uniformly supported these findings \cite{adam_hydration_2008, ely_hypohydration_2013}. The reason conflicting evidence is reported remains unclear; however, some potential variables include differences within methods to elicit HYPO (e.g., exercise, exercise-heat stress, fluid restriction, diuretics), the magnitude of HYPO (measured as change in body mass loss from 1 to 6\%), and the specific cognitive-motor task selection \cite{lieberman_hydration_2007, lieberman_methods_2012,masento_effects_2014}. While narrative reviews have addressed this topic \cite{masento_effects_2014, lieberman_methods_2012, grandjean_dehydration_2007}, a quantitative analysis is lacking that evaluates potential confounding variables which may explain the disparate results regarding the potential effects of HYPO on cognitive-motor performance. 

Although current literature is equivocal, one consistency is that extreme levels of HYPO (e.g., $>$8\% BM loss) have elicited cognitive-motor impairments \cite{king_brief_1878, adolf_physiology_1947}. Soldiers in adverse environments (e.g., desert heat with extended water restriction) have impaired ability to navigate, successfully complete military operations, and, if HYPO is severe enough, may present with confusion and delirium \cite{king_brief_1878,adolf_physiology_1947}. Soldiers undergoing 5\% body mass loss during a 72h training exercise had impaired (by 2-4 fold) vigilance, reaction time, attention, memory, and reasoning compared to their performance at rest \cite{lieberman_severe_2005}. However, these field-based military studies inducing large magnitudes of HYPO typically include factors known to alter cognitive-motor performance such as sleep deprivation \cite{krause_sleep-deprived_2017}, hypoglycemia \cite{strachan_acute_2001}, and other physiological stressors \cite{opstad_performance_1978}. However, even with these potential confounding factors, these studies suggest that 5\% BM loss is a critical threshold for cognitive impairments but the mechanism by which hypohydration alone could deteriorate cognitive-motor performance is not understood. 

There is no clear threshold established to determine at magnitude of HYPO (e.g., ${\geq}$  2\% BM loss) cognitive-motor impairments are observed  \cite{gopinathan_role_1988,sharma_influence_1986,sharma_differential_1983}. However, because 2\% BM loss elicits physical (e.g., aerobic) performance decrements \cite{sawka_hypohydration_2015} and physiological compensation (e.g., responses to increased plasma osmolality) \cite{cheuvront_dehydration:_2014}, some suggest cognitive-motor impairments followed in parallel. However, subsequent studies have not supported this direct relationship between HYPO and cognitive-motor impairments. 

Therefore, our purpose was to utilize quantitative analytical techniques (i.e., meta-analysis) to determine whether published evidence indicates if HYPO impairs cognitive-motor performance. Our primary aim was to examine potential design factors (e.g., method to elicit HYPO, magnitude of HYPO) that may influence the effect. We hypothesize that, similar to previous narrative reviews, HYPO will induce a small but significant impairment in cognitive-motor performance. Secondly, we hypothesize that HYPO magnitude will significantly correlate to magnitude of cognitive-motor performance impairment. Lastly, we hypothesize there will be a greater cognitive-motor impairment in those studies eliciting BM losses $>$2\% compared to with $<$2\%.
 
\section{Methods}
A systematic review was conducted on the research literature for the effects of HYPO and cognitive-motor performance. Cognitive-motor performance was operationally defined as any measurable outcome resulting from completion of a cognitive-motor function task (e.g., reaction time, accuracy). The literature search was last completed on September 2017. Searches were conducted in the following databases: PubMed, Medline, Psych Info, Sport Discuss, Web of Science, SCOPUS, ProQuest Theses and Dissertations, which collectively returned 8306 results (6591 without duplicates). References from relevant review articles were also examined \cite{lieberman_hydration_2007,masento_effects_2014,benton_small_2015,nuccio_fluid_2017} for articles not uncovered previously. Search terms to obtain studies relevant to this meta-analysis consisted of: ``dehydration'', ``hypohydration'', ``water loss'', ``hypovolemia'', ``hypovolemic'', ``sweat loss'', ``fluid restriction'', ``cognition'', ``cognitive function'', ``cognitive performance'', ``intelligence'', ``mental'', ``expertise'', ``recall'', ``executive function'', ``response time'', ``reaction time'', ``memory'', ``perception'', ``vigilance'', ``pattern recognition'', ``processing'', ``recognition'', ``sensory'', ``decision making'', ``attention'', and ``mood''. 

\subsubsection{Inclusion Criteria}
Studies meeting the following inclusion criteria were considered for review: i) the study was conducted on healthy (i.e., no clinical conditions) humans, ii) the study contained at least two measurements (in one or more sessions) within one or separate  groups of subjects where cognitive-motor testing was completed following HYPO and under control conditions , iii) changes in hydration status were reported using previously established biomarkers \cite{cheuvront_physiologic_2013}, and iv) cognitive-motor performance variable (e.g., accuracy, reaction time) was reported. Exclusion criteria included studies which implemented a prolonged HYPO intervention (${>}$ 48 h).     

\subsubsection{Selection of Studies}
A total of 6591 relevant publications were originally identified through the database searches. Of those, 6512 were initially excluded on the basis of title and/or review of the abstract (PRISMA diagram summarizes exclusion criteria in Figure \ref{fig:prisma}. Therefore, 79 publications underwent full-text examination for suitability in the current analysis. Of those 79, 48 were excluded, leaving a total of 31 articles to be included within the meta-analysis, all but one \cite{kakos_improving_2013} of which were published.

%%% PRISMA DIAGRAM
\begin{figure}
  	\centering
  	\includegraphics[width=12cm]{figures/"prisma_ma".pdf}
  	\caption{PRISMA diagram depicting the systematic review protocol in determining the population of studies for inclusion within the meta-analysis.}
  	\label{fig:prisma}

\end{figure}

\subsubsection{Data extraction}
All cognitive-motor tasks were included in the analysis. Each task was categorized as one cognitive-motor domain (e.g., executive funciton, working memory, attention) according to previously published criteria \cite{chang_effects_2012}. When specific cognitive-motor function classification was not immediately clear,  task categorization was determined by  author description. Continuous moderator variables were coded according to their actual value (e.g., magnitude of body mass loss) whereas categorical moderator variables (e.g., method to elicit HYPO) were coded with dummy variables corresponding to different levels of the variable. One primary moderator considered was the degree of HYPO. If not reported, BM measures were converted to a percent change score (\% Change = (BMpost - BMpre) / BMpost). Degree of HYPO was therefore examined as a moderator variable to assess whether the extent of body water losses (and resulting physiological perturbation) was related to alterations in cognitive performance. 

Methods to induce HYPO were coded into the following categories: exercise, heat exposure, exercise-heat stress (exercise + heat exposure), fluid restriction (${>}$ 12 h), or fluid restriction (${>}$ 12 h) + exercise/heat stress. The control condition within the study was also coded as being either rest (no exercise), exercise with fluid replacement, heat exposure with fluid replacement, or more than one control condition. If HYPO was induced using exercise, potential positive effects on cognitive-motor performance \cite{chang_effects_2012} were controlled for by coding the latency from end of exercise to cognitive testing (less than or $>$15 min). If information was provided about subject fitness level (both aerobic exercise testing or descriptive information), this information was captured by placing subjects into a category of either sedentary, low fit / unfit, moderately fit, fit / highly fit. Additionally, subject characteristics were coded for male, female, or both (percent male subjects). Cognitive-motor dependent variables were further grouped into either of two forms: reaction time or accuracy. For results not presented in these two specific formats, data was transformed (e.g., errors were considered a measure of accuracy, speed as reaction time).       

\subsubsection{Meta-Analysis}
The extracted cognitive-motor performance data were converted to a standard format by calculating the standardized mean change score or effect size (ES) using the metafor package for R (v1.9-9, www.metafor-project.org). Standardized mean change scores are beneficial in studies with psychological studies which have varying measurement scales \cite{morris_combining_2002,borenstein_introduction_2009}. In studies where the correlational data are not reported, r was estimated from the average of studies with reported HYPO-control correlation \cite{wittbrodt_fluid_2015,watson_mild_2015}, which was 0.62. For the effect size estimate, Hedge’s \textit{g} was employed to minimize the inherent bias of Cohen’s \textit{d} to overestimate the effect size when standardized mean differences are used with small samples sizes \cite{borenstein_introduction_2009}.   

Given the range of studies examining effects of HYPO on a variety of cognitive-motor performance measures, several dependent variables were available to code as outcomes measurements. Multiple effect sizes are problematic for most conventional meta-analyses, as the dependent structure of results (e.g., decreased reaction time but increased accuracy) may confound and compromise validity of the meta-analysis results unless the covariance structure of the results is known \cite{scammacca_meta-analysis_2014}. Given that no study in the selected studies presented this data, other meta-analytic methods were required. Recently, a robust variance estimation meta-analytical technique was developed which accounts for the within-study dependence by estimating the covariance matrix \cite{hedges_robust_2010}. This technique was therefore utilized in the current study within the robumeta package in R (\textit{cran.r-project.org/web/packages/robumeta}) using the \texttt{robu()} function. Furthermore, this technique has been recently refined to include small-sample corrections (${<}$ 40 studies) \cite{tipton_small_2015}, appropriate for the current analysis of 31 studies. The small-sample corrections utilize a calculated degrees of freedom using the Satterwaite approximation (df = 2 / cv${^2}$, where cv is the coefficient of variation) \cite{tipton_small_2015}. However, because degrees of freedom are calculated, any value ${\le}$ 4 represents an elevated Type I error risk (${>}$ 10\%) where results may not represent population values \cite{tipton_small_2015}. Rho (${\rho}$), the value accounting for dependencies of effect sizes was set at the recommended 0.8 \cite{tanner-smith_robust_2014}. Subsequent sensitivity analyses revealed that changes in ${\rho}$ (from 0 to 1) did not influence the overall effect size (maximum change = 0.0002) in the current study.

For all the analyses, a negative ES represents that HYPO impaired cognitive-motor performance whereas a positive ES represents an improvement. The robust variance estimation technique outputs the standardized mean difference between groups. Therefore, a significant covariate indicates that the effect size estimate depends on the value of that covariate.

\section{Results}

\subsection{Study Characteristics}
The final sample consisted of 31 studies (m), all but one \cite{kakos_improving_2013} was published. In total, there were 253 effect sizes (k; per study: min = 1, mean = 8.16, median = 6, max = 36) representing a total of 388 subjects. The characteristics of each study is presented in Table \ref{App:deh_cog}. 

\subsection{Primary Analyses}
Figure \ref{fig:forest_plot} presents effect sizes of all studies examining the effect of HYPO on cognitive-motor performance. Overall (m = 31, k = 253), HYPO elicited a small but significant impairment in cognitive-motor performance (g = -0.28, t(29.3) = -4.4, p = 0.0001, 95\% CI: [-0.41, -0.15]). There was substantial heterogeneity within studies (I${^2}$ = 68.60), however the between-study heterogeneity was acceptable (${\tau}$${^2}$ = 0.18). HYPO had a small (ES $<$ 0.2 - 0.5) but significant impairment on reaction time (m = 23, k = 76, g${^+}$ = -0.20, t(21.6) = -2.3, p = 0.03, 95\% CI: [-0.38, -0.02]) and accuracy based outcomes (m = 29, k = 177, g${^+}$ = -0.25, t(27.1) = -3.5, p = 0.002, 95\% CI: [-0.41, -0.10]).

When assessing the magnitude of HYPO (\% body mass loss), the significant overall cognitive performance impairment persisted (g${^+}$ = -0.28, t(27.93) = -4.3, p = 0.0002, 95\% CI: [-0.41, -0.14], Figure \ref{fig:overall_dehLevel}), although there was no significant association (slope = -0.03, t(5.2) = -0.70, p = 0.51). The significant overall main effect was also preserved when controlling for subject age (range: 19.1 - 34.0 y, g = -0.29, t(26.3) = -4.2, p = 0.0003, 95\% CI: [-0.43, -0.15]), although there was no significant association (slope = 0.013, t(5.7) = 0.20, p = 0.85). When examining methods of eliciting HYPO, exercise and exercise-heat stress were most commonly implemented and both elicited significant cognitive-motor impairments (p $<$ 0.05, Table \ref{tbl:moderators}). Lastly, when fitness level was measured/described, only recreationally or highly fit subjects were included. Both recreationally and highly fit individuals had impaired cognitive-motor performance following HYPO (p ${<}$ 0.05; Table \ref{tbl:moderators}).  

%%% FOREST PLOT
\begin{figure}
	\centering
	\includegraphics[width=13cm]{figures/"forest_plot".pdf}
	\caption{Forest plot of overall studies examining the effect of hypohydration on cognitive-motor performance. There was an overall significant main effect (g${^+}$ = -0.28, p = 0.0001). Negative effect sizes (ES) indicate that hypohydration impaired cognitive-motor performance whereas positive ES indicates improvement. Box size indicates relative weight attributed to each effect size within the study.}
	\label{fig:forest_plot}
\end{figure}

%%% MODERATOR TABLE
\begin{table}
	\caption{Moderator Analysis on the relationship of categorical variables on cognitive performance. ${^a}$df ${<}$ 4, potential Type I error rate ${>}$ 0.10; g${^+}$: effect size, SE: standard error of the effect size estimate; df: degrees of freedom; t: t test statistic for the covariate; p: p value for the t test, m: number of studies, k: number of outcomes}
	\centering
	
	\begin{tabular}{lcccccc} 
		\hline
		\textbf{Variable} & \textbf{g${^+}$} & \textbf{SE} & \textbf{df} & \textbf{p} & \textbf{m} & 
		\textbf{k} \\
		\hline
		\textit{Method of Dehydration} &&&&&& \\
		Exercise & -0.29 & 0.12 & 6.0 & \textbf{0.05} & 15 & 46 \\
		Heat Stress & -0.22 & 0.14 & 4.0 & 0.19 & 5 & 31 \\
		Exercise-Heat Stress & -0.22 & 0.10 & 14.6 & \textbf{0.04} & 16 & 177 \\
		Fluid Restriction & -0.50 & 0.15 & 2.1${^a}$ & 0.07 & 4 & 9 \\
		Fluid Restriction + EHS	& -0.18 & 0.06 & 1.0${^a}$ & 0.19 & 2 & 7 \\
		&&&&&& \\
		\textit{Subject Fitness} &&&&&& \\
		Recreationally Fit & -0.28 & 0.11 & 11.5 & \textbf{0.02} & 13 & 126 \\
		Highly Fit &  -0.32 & 0.10 & 10.9 & \textbf{0.01} & 12 & 82 \\ 
		\hline	    
	\end{tabular}
	\label{tbl:moderators}
\end{table}

\subsection{Executive Function}
Figure \ref{fig:ex_func_ma} presents the study-specific and overall results for tasks of executive functions assessed following HYPO. Executive functions represented a subset of the overall data set limited to tasks including executive control, working memory, and information processing \cite{shields_does_2015}. The final sample consisted of 21 studies (m), yielding a total of 93 outcomes (k, range: 1 - 18 per study, mean = 2.36), with all data from published manuscripts. Overall, HYPO had a small, but significant, impairment on executive functions (m = 21, k = 93, g${^+}$ =  -0.24, t(19.5) = -2.7, p = 0.01, 95\% CI [-0.43, -0.06]). There was large heterogeneity between-studies (I${^2}$ = 69.2) but acceptable within-study heterogeneity (${\tau}$${^2}$ = 0.20). HYPO significantly impaired accuracy based outcomes within tests of executive functions (g${^+}$ = -0.29, t(18.3) = -2.8, 95\% CI: [-0.52,-0.07], p = 0.01). However, reaction-time based outcomes did not appear to be affected (g${^+}$ = -0.02, 95\% CI: [-0.34, 0.29], p = 0.87).

When controlling for the magnitude of HYPO (mean = 2.3\% BM loss), the significant impairment in executive function was unaffected (g${^+}$ =  -0.19, p = 0.02, Figure \ref{fig:ex_func_bm_correlation}), however, there was no significant association was present (slope = -0.06, 95\% CI: [-0.21, 0.09], p = 0.41). A similar pattern was also observed for subject age (g${^+}$ = -0.19, 95\% CI: [-0.35, -0.03], p = 0.02), with no relationship to cognitive-motor impairment (slope = 0.03, 95\% CI: [-0.14, 0.19], p = 0.73). When examining categorical moderators, no method to induce HYPO significantly impaired executive function performance (p ${>}$ 0.05, Table \ref{tbl:ex_func_moderators}). However, it does appear that moderately subjects had greater executive function performance impairments following HYPO (p ${<}$ 0.05) compared to highly fit subjects (Table \ref{tbl:ex_func_moderators}).

%%% EXECUTIVE FUNCTION FOREST PLOT
\begin{figure}
	\includegraphics[width=14cm]{figures/"ex_function_forest".pdf}
	\caption{Forest plot of overall studies examining the effects of hypohydration on executive functions. The overall effect was not significant (g${^+}$ = -0.24, p = 0.01). Negative effect sizes (ES) indicate that hypohydration impaired executive functioning whereas positive ES indicates improvement. Box size indicates relative weight attributed to each effect size within the study.}
	\label{fig:ex_func_ma}
\end{figure}

%%% EX FUNCTION MODERATOR TABLE
\begin{table}
	\caption{Moderator analysis for variables contained within studies examining executive functioning following hypohydration. ${^a}$df ${<}$ 4 (Type I error rate ${>}$ 0.10); g${^+}$: effect size, SE: standard error of the effect size estimate; df: degrees of freedom; t: t test statistic for the covariate; p: p value for the t test, m: number of studies, k: number of outcomes}
	\centering
	\begin{tabular}{lcccccc} 
		\hline
		\textbf{Variable} & \textbf{g${^+}$} & \textbf{SE} & \textbf{df} & \textbf{p} & \textbf{m} & 
		\textbf{k} \\
		\hline
		\textit{Method of Hypohydration} &&&&&& \\
		Exercise & -0.16 & 0.15 & 3.0${^a}$ & 0.37 & 4 & 8 \\
		Heat & -0.21 & 0.14 & 4.0 & 0.19 & 5 & 20 \\ 
		Exercise-Heat Stress & -0.23 & 0.13 & 10.8 & 0.11 & 12 & 61 \\
		&&&&&& \\
		\textit{Subject Fitness} &&&&&& \\
		Moderately Fit & -0.34 & 0.15 & 9.7 & \textbf{0.05} & 11 & 54 \\
		Highly Fit & -0.18 & 0.12 & 4.0 & 0.22 & 5 & 17 \\
		\hline		    
	\end{tabular}
	\label{tbl:ex_func_moderators}
\end{table}

\subsubsection{Studies with ${>}$2\% BM Loss}
Figure \ref{fig:two_deh_forest_plot} presents a summary of studies assessing HYPO of ${\leq}$2\% BM. Out of the total dataset of 31 studies, only 58\% elicited enough HYPO to elicit physiological changes \cite{cheuvront_dehydration:_2014}. The final sample consisted of 18 studies (m), yielding a total of 132 outcomes (k, range: 1 - 18 per study, mean = 7.3), with all data being from published manuscripts. The overall effect of HYPO on cognitive-motor performance on studies eliciting ${\geq}$2\% BM loss was small, but significant, g${^+}$ =  -0.31, t(16.5) = -3.2, p = 0.006, 95\% CI [-0.51, -0.10]. Within this subset of studies, accuracy-based outcomes were significantly impaired by HYPO (m = 18, k = 97, g${^+}$ = -0.36, 95\% CI: [-0.59, -0.13], p = 0.005), however, reaction time based outcomes were not (m = 14, k = 35, g${^+}$ = -0.15, 95\% CI:[-0.37, 0.08]).

When controlling for magnitude of HYPO, the overall significant impairment in cognitive-motor performance persisted (g${^+}$ = -0.30, 95\% CI: [-0.51, -0.09], p = 0.007, Figure \ref{fig:deh_2_dehLevel}), however, there was no significant association was observed (slope = -0.03, p = 0.71). When controlling for subject age, the overall significant impairment in cognitive-motor performance persisted (g${^+}$ = -0.32, 95\% CI: [-0.55, -0.10], p = 0.008), however, there was no significant relationship was observed (slope: 0.06, p = 0.48). Table \ref{tbl:ehs_2_moderators} presents other moderator variables within this subset of studies. While there did appear to be a significant cognitive-motor impairment following HYPO of ${\geq}$2\% BM loss induced by exercise (p = 0.05), the estimated degrees of freedom (df = 2.9) indicates a ${>}$10\% Type I error rate, and therefore the result cannot be trusted. Lastly, it appears that the highly fit subjects were more susceptible to cognitive-motor impairments following HYPO to a magnitude of HYPO ${\ge}$2\% BM loss.   

%%% 2% MODERATORS
\begin{table}
	\caption{Moderator analysis for studies eliciting at least 2\% body mass loss. ${^a}$df ${<}$ 4 (Type I error rate ${>}$ 0.10); g${^+}$: effect size, SE: standard error of the effect size estimate; df: degrees of freedom; t: t test statistic for the covariate; p: p value for the t test, m: number of studies, k: number of outcomes}
	\centering
	\begin{tabular}{lcccccc} 
		\hline
		\textbf{Variable} & \textbf{g${^+}$} & \textbf{SE} & \textbf{df} & \textbf{p} & \textbf{m} & 
		\textbf{k} \\
		\hline
		\textit{Method of Hypohydration} &&&&&& \\
		Exercise & -0.27 & 0.09 & 2.9${^a}$ & \textbf{0.05} & 4 & 20 \\
		Heat Stress & -0.114 & 0.065 & 1.9${^a}$ & 0.23 & 3 & 11 \\
		Exercise-Heat Stress & -0.34 & 0.20 & 7.9 & 0.12 & 9 & 79 \\
		&&&&&& \\
		\textit{Subject Fitness} &&&&&& \\
		Moderately Fit & -0.53 & 0.25 & 4.9 & 0.09 & 6 & 45 \\
		Highly Fit & -0.28 & 0.12 & 6.7 & \textbf{0.05} & 8 & 55 \\ 
		\hline		    
	\end{tabular}
	\label{tbl:ehs_2_moderators}
\end{table}


%%% 2% FOREST PLOT
\begin{figure}
	\centering
	\includegraphics[width=10cm]{figures/"two_deh_forest_plot".pdf}
	\caption{Forest plot of overall studies utilizing exercise-heat stress to induce hypohydration and eliciting ${\ge}$ 2\% body mass loss. The overall effect was not significant (g${^+}$ = -0.337, p = 0.112). Negative effect sizes (ES) indicate that hypohydration impaired executive functioning whereas positive ES indicates improvement. Box size indicates relative weight attributed to each effect size within the study.}
	\label{fig:two_deh_forest_plot}
\end{figure}


\subsubsection{Visuomotor and Occupation Specific Performance}
Figure \ref{fig:mc_forest} presents studies examining visuomotor/occupational specific performance following HYPO. The final sample consisted of five studies (m = 6), yielding a total of 15 outcomes (k, range: 1 - 6 per study, mean = 2.5), all from published manuscripts. Overall, HYPO did not impair visuomotor performance (g${^+}$ = -0.30, t(4.7) = -1.6, p = 0.17, 95\% CI [-0.79, 0.19]). When controlling for the magnitude of HYPO, the overall effect size was reduced (g${^+}$ = 0.02, 95\% CI: [-2.5, 2.6], p = 0.98, Figure \ref{fig:mc_dehLevel}), although there was no significant association (slope = -0.13, p = 0.70). Within this subset of studies it was not possible to examining moderators, as the paucity of data led to df calculations $<$4, increasing the Type I error rate $>$10\%.     

\begin{figure}
	\centering
	\includegraphics[width=10cm]{figures/"mc_forest".pdf}
	\caption{Forest plot of overall studies examining the effects of hypohydration on visuomotor/occupational-specific performance. The overall effect was not significant (g${^+}$ = -0.30, p = 0.17). Negative effect sizes (ES) indicate that hypohydration impaired executive functioning whereas positive ES indicates improvement. Box size indicates relative weight attributed to each effect size within the study.}
	\label{fig:mc_forest}
\end{figure}

\section{Discussion}
Hypohydration is believed to impair cognitive-motor performance and potentially increase workplace accidents and occupational risk \cite{kenefick_hydration_2007}. While various narrative reviews have suggested HYPO may impair cognitive-motor performance, previous research has not unanimously supported this position \cite{cheuvront_dehydration:_2014}. To that extent, we instituted a quantitative analysis of the previous literature using a meta-analysis to understand the objective effect of HYPO on cognitive-motor performance. The main finding of this study is that HYPO had a small but significant negative effect on cognitive-motor performance. However, the level of impairment does not appear to be associated with magnitude of HYPO.

The small, but significant, negative effect of HYPO on cognitive-motor performance aligns with some narrative reviews suggesting HYPO may mirror the effects of other nutritional interventions \cite{masento_effects_2014}. Interestingly, both accuracy and reaction-time based outcomes were significantly impaired by HYPO. Accuracy-related outcomes during cognitive-motor testing (i.e., errors, percentage correct) likely represent cognitive-processing faced within various occupational scenarios \cite{wickens_multiple_2002}, and therefore may indicate judgment/decision making processes are impaired following HYPO. The mechanism explaining why accuracy-based outcomes would be impaired following HYPO are not fully understood, although some have demonstrated increased neural activation (i.e., inefficient processing) following HYPO \cite{kempton_dehydration_2011}, leading to inefficient cognitve-motor processing which could result in errors. In contrast, reaction time is largely influenced by the visual identification of a stimulus combined with a motor response. Multiple reports have observed no impairment to visual discrimination \cite{grego_influence_2005, van_den_heuvel_independent_2017} or maximal finger tapping frequency \cite{bandelow_effects_2010} following HYPO, potentially suggesting resiliency within cognitive-motor systems stressed during reaction time measurements. However, our observed impairment contradicts these mechanisms, and suggests a slowed cognitive-motor processing stream following HYPO.

To further probe the literature, subsets of the database were isolated to quantitatively examine potential confounding variables hypothesized to influence how HYPO may impair cognitive-motor performance. Controlling for HYPO magnitude (expressed as a percent of body mass loss) did not alter the small, but significant, overall impairment in cognitive-motor performance. However, two factors may have influenced these findings: i) the small overall ES which truncates ability to observe graded responses and ii) minimal studies which induced multiple magnitudes of HYPO. In one such study \cite{gopinathan_role_1988}, greater levels of HYPO induced greater cognitive-motor impairments across the domains of crystallized intelligence, executive function, and short-term memory. While these within-study designs offer superior control over various confounding variables, not all studies have observed marked changes, with some observing no significant exacerbation of cognitive-motor impairment with greater levels of HYPO \cite{weber_dehydration_2013, baker_dehydration_2007}. The second primary moderator variable examined was whether the method to elicit HYPO impacted cognitive-motor responses. It appears that, when HYPO is induced via exercise-heat stress or exercise alone, cognitive-motor performance is significantly impaired. This agrees with ideas presented previously that methods to induce HYPO may contribute to the equivocal literature \cite{lieberman_methods_2012}, although how this manifests itself has not been previously understood. One interesting observation was the fluid restriction had a larger computed impairment on cognitive-motor performance (g${^+}$ = 0.50), although the Satterthwaite approximations \cite{tipton_small_2015} resulted in a degrees of freedom level inducing a Type I error rate ${>}$ 10\%, likely as a result from a paucity of available data (4 studies, 9 effects). Therefore, more research is needed to provide ample resolution of this potential impairment in cognitive-motor performance following fluid restriction protocols.    

The second important finding of this study was that executive functions (e.g., executive function, working memory, information processing) were impaired to a small but significant amount following HYPO. Various studies have suggested higher-order cognitive-motor components may be disproportionately impaired by HYPO \cite{tomporowski_effects_2007, lieberman_hydration_2007,grandjean_dehydration_2007,nuccio_fluid_2017}, although the proposed mechanism has not been throughly presented. HYPO has previously increased neural resource requirements during an executive function test despite no impairment in performance in adolescents \cite{kempton_dehydration_2011}. The observed impairments were located within the frontal and parietal lobes \cite{kempton_dehydration_2011}, areas integral to executive functioning \cite{logue_neural_2014}. However, because executive function impairments can result from deficits in (at least) one of the cognitive processes required for task completion (e.g., inhibition, attention) \cite{logue_neural_2014}, why, exactly, impairments occur following HYPO is unknown. Some have suggested HYPO may manifest in mood changes leading to impaired cognitive-motor performance \cite{masento_effects_2014}, however, it should be noted changes to mood and executive function performance may not be linked, especially in real life scenarios \cite{lagner_mood_2015}. Interestingly, no method to elicit HYPO impaired HYPO to a greater extent, which conflicts with some narrative reviews \cite{nuccio_fluid_2017}. However, it should be noted that the sample for some methods (exercise, heat stress) is limited (m ${\leq}$ 5), which may have potentially limited conclusions when coupled with the large known between-study heterogeneity.     

Another finding of these study was that studies eliciting a magnitude of HYPO required to elicit physiological compensation \cite{cheuvront_dehydration:_2014} had a larger cognitive-motor impairment than the population of studies by ${\sim}$50\%. Interestingly, a relatively small proportion of studies elicited this magnitude of HYPO, as previous research is robust in supporting 2\% BM loss is required to be ${>}$95\% confident of altered body water status \cite{cheuvront_biological_2010} and the robust effects on aerobic performance \cite{sawka_hypohydration_2015, sawka_american_2007}. Furthermore, many of the previous studies \cite{wittbrodt_exercise-induced_2015,weber_dehydration_2013} eliciting $<$2\% BM loss did not include a robustly established baseline hydration status to ensure a hypohydrated state at these modest body water losses \cite{cheuvront_dehydration:_2014,cheuvront_daily_2004}. Furthermore, it does not appear that method to induce HYPO of 2\% BM loss impacts cognitive-motor responses. Interestingly, of the 18 studies included within this subset, three reported improved cognitive-motor performance following 2\% HYPO, although none had 95\% confidence intervals being exclusively positive \cite{van_den_heuvel_independent_2017,kakos_improving_2013,tomporowski_effects_2007}. One commonality within these three studies \cite{tomporowski_effects_2007,kakos_improving_2013} is cognitive-motor assessment immediately following HYPO intervention when exercise-mediated improvements in cognitive-motor function are present \cite{chang_effects_2012}. Furthermore, within one study \cite{tomporowski_effects_2007}, both HYPO trials were averaged to present the reported decrement in executive function accuracy, and multiple data points were presented on the areas which had an improvement.  

In contrast with the other results presented, there was no significant effect of HYPO on visuomotor/occupation-specific performance. it should be noted that this meta-analysis did not include studies without data suited for inclusion in the meta-analysis, including studies finding an overall improvement \cite{bandelow_effects_2010} or no change \cite{cian_effects_2001, hogervorst_cognitive_1996} in maximal button tapping speed following HYPO and another finding flight simulation impairments but without reported BM changes \cite{lindseth_effects_2013}. The included studies utilized many different facets of visuomotor performance such as motor coordination, stability, or gross-motor task \cite{turner_mild_2017}. How each of these respond to HYPO requires further research.

Lastly, although not included within this study, HYPO-mediated cognitive-motor impairments may result from affective changes such as altered mood \cite{armstrong_mild_2012,masento_effects_2014}, increased feeling of mental exertion \cite{szinnai_effect_2005}, or the presence of thirst \cite{egan_neural_2003}. Worsened mood has been observed previously to occur both at modest levels of HYPO (${\sim}$1\% BM loss) \cite{ganio_mild_2011,armstrong_mild_2012} and without concomitant cognitive-motor impairment \cite{armstrong_mild_2012}. Secondly, mental exertion may be elevated during cognitive-motor testing \cite{szinnai_effect_2005}, however, this may not impair performance. While neuroimaging data suggests perceived elevated mental exertion occur with increased brain activation \cite{kempton_dehydration_2011,watson_mild_2015}, the exact influence on this influencing cognitive-motor impairments is not clear. The affective factor with the greatest potential impact to cognitive-motor performance could be thirst. This is evidenced by HYPO magnitude correlating to thirst sensation severity \cite{engell_thirst_1987} while activations within the anterior cingulate cortex respond to the presence/quenching of thirst \cite{egan_neural_2003}. Thirst may also be related to elevated cortisol levels following HYPO \cite{mcmorris_heat_2006} and, because cortisol provides negative feedback from vasopressin release from the paraventricular nucleus \cite{andreoli_endocrine_2010}, elevated plasma concentrations indicate hypothalamus-pituitary-adrenal axis activation \cite{lieberman_severe_2005}. Therefore, these factors may potentially influence HYPO-mediated cognitive-motor impairments, however, the mechanisms/contributions are unclear. 

In conclusion, we have identified that HYPO elicits a small, but significant impairment in cognitive-motor performance. The impairment in cognitive-motor performance appears to appear in both accuracy and reaction-time based outcomes, and is not associated with magnitude of HYPO. This cognitive-motor impairment is observed specifically during tasks of executive functions and in studies eliciting ${\geq}$2\% body mass loss, however, the magnitude was similar to the overall analysis. The specific influence of affective factors may contribute to the cognitive-motor impairment, however, this requires further investigation. 
 