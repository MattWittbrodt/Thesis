\chapter{Introduction and Background}
There are reasons to believe body water deficits (hypohydration; HYPO) can have adversely impact central nervous system (CNS) function. Severe levels of HYPO (e.g., prolonged water restriction in the desert) elicit symptoms of CNS dysfunction such as decreased mental alertness, personality changes (e.g., hostility), and headaches \cite{adolf_physiology_1947, king_brief_1878}. However, at more modest states of HYPO, the effects on the CNS are less clear. For example, certain populations (e.g., elderly) may be at greater risk for confusion and delirium with HYPO, leading to increased frequency, duration, and severity of hospitalizations \cite{inouye_delirium_2006,warren_burden_1994}. Furthermore, small to moderate levels of HYPO may also degrade human-system performance reliant on adequate CNS function, as evidenced by impaired driving ability \cite{watson_mild_2015} and performance during occupation-specific simulations (e.g., pilot proficiency) \cite{lindseth_effects_2013}. Indirect evidence also suggests occupational accident rates are increased during hot weather/summer months when HYPO is likely \cite{kenefick_hydration_2007, kenefick_dehydration_2012}. Lastly, some have suggested that the acute act of drinking water (when marginal body water deficits of ${<}$ 1\% body mass (BM) loss) can improve memory in both school children and adults \cite{benton_effect_2009,benton_small_2015,benton_minor_2016,fadda_effects_2012}. However, despite these numerous observations, the published peer-reviewed literature remains inconsistent regarding whether moderate HYPO impairs CNS function \cite{ely_hypohydration_2013,adam_hydration_2008,danci_voluntary_2009}, which CNS functions may be most susceptible to HYPO, and what magnitude of HYPO can be tolerated before decrements in cognitive task performance occur.


%%%%%%%%%%%%%%%%%%%%%%%%%%%%%
%%%%% Physiology of DEH %%%%%
%%%%%%%%%%%%%%%%%%%%%%%%%%%%%

\section{Physiological Responses to Hypohydration}
Hypohydration elicits severe consequences to physiological homeostasis \cite{iom_dietary_2004}. In the body, water is stored within three compartments: plasma (8\% of total body water), interstitial fluid (25\%), and intracellular fluid (67\%) (Figure \ref{fig:fluidVolumes}) \cite{sawka_thermoregulatory_1985}, with water transfer among these compartments largely governed by Starling forces (capillary and/or osmotic exchange). The composition of fluid lost (e.g., sweating or diarrhea) influences the osmotic gradient and, therefore, dictates physiological responses to HYPO. Because these physiological responses vary in their sensitivity (i.e., greater body water loss increases up-regulation of response mechanism), HYPO is classified as either intracellular or extracellular \cite{cheuvront_dehydration:_2014}. For the purpose of this thesis, only intracellular HYPO will be considered. Intracellular HYPO occurs via water losses through non-illness (e.g., diarrhea) or non-medicinal (e.g., diuretic) means such as exercise, heat stress, or fluid restriction. 

\begin{figure}
	\centering
	\includegraphics[width=8cm]{figures/"Fluid Volumes".pdf}
	\caption{Distribution of body water among compartments. TBW = Total Body Water. \textit{Adapted from \cite{sawka_thermoregulatory_1985}}.}
	\label{fig:fluidVolumes}
\end{figure}

As previously stated, HYPO elicited by exercise-heat stress (EHS) results in intracellular hypohydration. During EHS, elevations in sweat rate (a byproduct of increased metabolic heat production from the exercising muscles) \cite{sawka_integrated_2011} results in marked losses of sweat, which is hypotonic fluid to blood plasma (i.e., less solute for given volume). Therefore, more relative water is lost compared to solute, which results in an osmotic gradient between the blood plasma and interstitial fluid. The blood plasma becomes hypertonic, which, via Starling forces, draws water from the interstitial compartment into the blood plasma and from the intracellular to interstitial compartment \cite{cheuvront_dehydration:_2014}. When exracellular fluid osmolality increases by 2\%, CNS osmoreceptors for anti-diuretic hormone and angiotensin II and thirst depolarize \cite{bourque_central_2008}, resulting in water conservation and drive for water acquisition \cite{andreoli_endocrine_2010}. Intracellular HYPO is also termed hypertonic hypovolemia to describe blood volume losses which are coupled coupled with a hypertonic plasma \cite{darrow_changes_1935}. 

Resulting from the previously described physiological perturbations, HYPO impairs aerobic exercise performance. While most data suggests HYPO of ${\geq}$2\% body mass (BM) loss is required to observe impairments \cite{sawka_hypohydration_2015}, degradations have been reported at -1\% BM loss \cite{bardis_mild_2013}. Aerobic exercise performance (as measured by time trial or time to exhaustion tests) appears to be more robustly examined in warm/hot environments \cite{sawka_human_1992,cheuvront_mechanisms_2010,ebert_influence_2007}, likely as a result of exacerbated cardiovascular strain \cite{costill_plasma_1974,cheuvront_mechanisms_2010} and high skin temperature \cite{sawka_high_2012}. When HYPO is coupled with a skin temperature of ${\sim}$27${^o}$C, submaximal aerobic performance is impaired, with 1.3\% performance drops with every 1${^o}$C \cite{sawka_high_2012}. This is also suggested to impair intermittent exercise, as observed with decreases to physical performance in team sports (e.g., soccer) \cite{nuccio_fluid_2017}.     

While the effects of aerobic performance are clear, the impact of HYPO on muscular strength is less convincing \cite{cheuvront_dehydration:_2014}. For example, a recent review article found almost unanimous agreement that HYPO impairs aerobic endurance performance at ${\geq}$4\% BM loss (19/21 studies or 90\%), but only 19\% (25/133 studies) observed impairments in muscular power \cite{cheuvront_dehydration:_2014}. Some muscular power measurements may not be sensitive to the effects of HYPO, however, as evidence by decreased ground reaction forces following 3\% BM loss without any changes in vertical jump height \cite{cheuvront_hypohydration_2010}. It may also be that muscular endurance is impacted to a greater extent than power, as demonstrated by shortened time to fatigue during submaximal isometric quadriceps contraction (25\%) without change to maximal voluntary contraction \cite{bigard_effects_2001}. These muscular endurance impairments occurred without metabolic changes within the muscle, and thus a mechanism was not established \cite{bigard_effects_2001}. Lastly, a meta-analysis \cite{savoie_effect_2015} observed degraded performance of 5-8\% on multiple components of muscular performance (e.g.,endurance, strength, and anaerobic power).

%%%%%%%%%%%%%%%%%%%%%%%%%%%%%
%%%%% DEH Brain Anatomy %%%%%
%%%%%%%%%%%%%%%%%%%%%%%%%%%%%

\section{Hypohydration and Brain Anatomy}
Until recently, the ability to image and analyze an \textit{in vivo} human brain has not been available. As a result, the first usage of this technology to scan the human brain following HYPO occurred in 2005 \cite{dickson_effects_2005, duning_dehydration_2005}. However, these two studies found different conclusions, with one observing finding no differences in brain volume following HYPO \cite{dickson_effects_2005}, but the other finding a significant decrease in total brain volume \cite{duning_dehydration_2005} (Table \ref{tbl:mri_deh}). Following these initial studies, others have also investigated brain responses elicited by body water losses, although the literature remains equivocal and present difficulty definitive understanding of how the human brain responds to HYPO. 

A total of eight studies have investigated the effects of HYPO on brain morphology (Table \ref{tbl:mri_deh}), with most reporting changes in total brain volume and/or lateral ventricle size. Only one study has observed decreases in total brain volume following HYPO, with a 16h fluid restriction protocol (1.6 \% body mass (BM) loss) decreasing brain volume by ${\sim}$0.5\%, although it should be noted individual responses ranged from +0.2 to -2.4\% \cite{duning_dehydration_2005}. The authors attributed decreased brain volume to the osmotic gradient induced by hypertonic hypovolemia shrinking the astrocytes which compromised fluid movement within the brain \cite{duning_dehydration_2005}. However, every other study has observed no changes in total brain volume with HYPO ranging from 0.8 - 2.9 \% body mass (BM) loss \cite{watson_effect_2010,kempton_effects_2009,kempton_dehydration_2011,meyers_does_2016}, suggesting this finding may be aberrant with regards to the literature at large.

Secondly, given the association between fluid balance and fluid-containing centers of the brain, many studies have attempted to determine changes within the lateral ventricles following HYPO. However, unlike total brain volume, the data on this topic is much less clear. While some studies have observed no significant difference \cite{meyers_does_2016, dickson_effects_2005, streitburger_investigating_2012,meyers_does_2016,dickson_effects_2005}, others have reported either an increase \cite{kempton_dehydration_2011,kempton_effects_2009} or decrease \cite{watson_effect_2010} in lateral ventricular volume following HYPO. Furthermore, some have suggested that lateral ventricle volume expansion significantly correlates to BM loss \cite{kempton_dehydration_2011, dickson_effects_2005}, suggesting magnitude of body water loss could elicit a graded increase in lateral ventricular volume. However, in both instances, these correlations were observed over a small range of HYPO (${\sim}$1\%), individuals with both ventricular expansion and shrinkage within the sample, and there was wide dispersion of ventricular volume changes (${\ge}$ 14\%). 

A common belief has been that ventricular expansion occurs from shrinking astrocytes limiting fluid movement throughout the brain \cite{kempton_dehydration_2011}. However, how other brain structures besides the ventricles change following HYPO has rarely been studied. Streitburger et al. \cite{streitburger_investigating_2012} found decreases in both white and grey matter in areas with close proximity to the brain ventricles, however, this was when hyperhydration was compared to HYPO. Although these are intriguing findings, the small sample size (n = 6) and minimal level of HYPO (-2.3\%) may be limiting factors. Furthermore, the voxel-based morphology approach did not allow for the analysis of individual brain structures, and therefore any relationship to changes in brain structures and changes in brain function cannot be established. Therefore, this thesis attempted to increase both sample size, magnitude of HYPO, and the parcellation of individual brain areas to better investigate the effects of HYPO on brain structures. 

%%%%%%%% Imaging Table
\begin{table}
	\caption{Previous Studies Using MRI to Investigate Hypohydration (HYPO) on Brain Morphology. EHS = Exercise Heat Stress, ND = No Difference (from control), CSF = cerebrospinal fluid, FR = Fluid Restriction. * Different versus a hyperhydration condition.}
	\begin{center}
		
		\begin{tabular}{lcccc} 
			\hline
			\footnotesize\textbf{Citation} & \footnotesize\textbf{Subject n} & \footnotesize\textbf{Method of HYPO \& Control} & \textbf{${\Delta}$ BM} & \footnotesize\textbf{Findings} \\
			\hline
			
			\footnotesize Meyers et al. \cite{meyers_does_2016} & \footnotesize 20 \scriptsize(11 M) & \footnotesize Fluid Restriction & \footnotesize -0.8\% & \footnotesize ND \\
			
			\footnotesize Duning et al. \cite{duning_dehydration_2005} & \footnotesize 10 \scriptsize(3 M) & \footnotesize FR; Pre-FR & \footnotesize -1.6\% & \footnotesize ${\downarrow}$ Brain Volume \\
			
			\footnotesize Kempton et al. \cite{kempton_dehydration_2011} & \footnotesize 10 \scriptsize(5 M) & \footnotesize EHS; EHS with fluid & \footnotesize -1.6\% & \footnotesize ${\uparrow}$ \footnotesize Lateral Ventricle \\
			
			\footnotesize Kempton et al. \cite{kempton_effects_2009} & \footnotesize 7 \scriptsize(7 M) & \footnotesize EHS; Pre-EHS & \footnotesize -2.2\% & \footnotesize ${\uparrow}$ \footnotesize Lateral Ventricle \\
			
			\footnotesize Dickson et al. \cite{dickson_effects_2005} & \footnotesize 6 \scriptsize(6 M) & \footnotesize EHS; Pre-EHS & \footnotesize -2.3\% & \footnotesize ND \\
			
			\footnotesize Streitburger et al. \cite{streitburger_investigating_2012} & \footnotesize 6 \scriptsize(3 M) & \footnotesize FR; Pre-FR & \footnotesize -2.3\% & \footnotesize ${\downarrow}$ Grey Matter*, ${\downarrow}$ White Matter* \\
			
			\footnotesize Watson et al. \cite{watson_effect_2010} & \footnotesize 8 \scriptsize(8 M) & \footnotesize EHS; Pre-EHS & \footnotesize -2.9\% & \footnotesize ${\downarrow}$ Total Ventricle \& CSF \\
			
			\footnotesize Nakumura et al. \cite{nakamura_correlation_2014} & \footnotesize 14 \scriptsize(12 M) & \footnotesize FR; Pre-FR \scriptsize(2wk prior) & \footnotesize ? & \footnotesize ND \\	
			\hline
			
		\end{tabular}
		\label{tbl:mri_deh}
	\end{center}
\end{table}


\section{Cellular Mechanisms of Central Water Homeostasis}
Given the observed changes to some brain areas following HYPO, questions have arisen about the potential mechanism resulting in findings. Although osmoregulation occurs via activation of both peripheral and central osmoreceptors \cite{bourque_central_2008}, this thesis will focus primarily central mechanisms of water homeostasis. The primary central osmoreceptors are located in brain areas without presence of a blood-brain barrier, the circumventricular organs, specifically the \textit{organum vasculosum laminae terminalis} (OVLT), which sits at the base of the third ventricle \cite{bourque_central_2008}. As the OVLT is exposed to hypertonic fluid, the cellular volume shrinks (intracellular HYPO), leading to a depolarized state of transient receptor potential vanilloid 1 receptors \cite{ciura_hypertonicity_2011} and increased action potential encoding \cite{bourque_central_2008}. Secondly, although not strictly osmoreceptors, many cellular types show osmoreceptive properties. Glial cells, specifically astrocytes, aid in regulating brain water and ion homeostasis in addition to maintaining the blood brain barrier are involved in the maintenance of the blood–brain barrier \cite{simard_neurobiology_2004}. Following the sensing of hypertonicity within the blood, arginine vasopressin (AVP) is released both systemically (to the body) and centrally (to the brain) \cite{simard_neurobiology_2004}. AVP therefore acts upon astrocytes to accomplish two things: i) influx of Na${^+}$, K${^+}$, and Cl${^-}$, ii) over-expression of aquaporin 4 to allow influx of water \cite{simard_neurobiology_2004}.   

These osmoprotective mechanisms are particularly robust within the brain, which maintains total volume despite large body water losses \cite{nose_distribution_1983,cipolla_cerebral_2009}. This likely results from neuroprotective mechanisms to maintain water content in brain tissue during osmotic stress such as ion pumps and leak channels \cite{cipolla_cerebral_2009,cserr_extracellular_1991}. The brain, unlike other body tissues, does not allow water influx by Starling forces, as the presence of high electrical resistance tight junctions (${\sim}$100x greater than capillaries) within the blood brain barrier (BBB) prevent movement of electrolytes and other hydrophilic substances \cite{kimelberg_water_2004,cipolla_cerebral_2009,ropper_hyperosmolar_2012,bain_cerebral_2015}. These protective mechanisms are also bolstered by astrocytes, which assist in maintaining structural integrity of the tight junctions, regulating ionic homeostasis, and buffering K${^+}$ \cite{cipolla_cerebral_2009}. Secondly, the brain produces organic osmolytes (myo-inositol, betaine, taurine, sorbitol) which help preserve cellular volume by increasing ionic concentration within the cell, equilibrating the osmotic gradient with the hypertonic exterior \cite{de_petris_cell_2001,weed_pressure_1919}. During chronic hypertonicity (extracellular osmolality greater than intracellular osmolality), organic osmolytes accounted for 35\% of brain tonicity (with no change in brain water content) \cite{lien_effects_1990} and were up-regulated by 44\% in an infant suffering severe HYPO \cite{lee_organic_1994}. However, because organic osmolyte gene encoding occurs following sensing of hypertonicity, appearance within the brain tissue is delayed \cite{de_petris_cell_2001, gullans_control_1993}. As a result, osmolyte-mediated equilibration of neuronal water content does not occur immediately following acute exposure to hypertonic environment \cite{ayus_effects_1996}. Therefore, although humans may be protected against large brain water perturbations, the delayed time course of osmolyte production may result in vulnerabilities to brain water homeostasis during acute HYPO as elicited by exercise-heat stress. 

%%%%%%%%%%%%%%%%%%%%%%%%%%%%%
%%%%%   DEH-Cognition   %%%%%
%%%%%%%%%%%%%%%%%%%%%%%%%%%%%

\section{Cognitive-Motor Responses to Hypohydration}
As previously stated, psychological/cognitive-motor impairments were observed in soldiers under severe physical distress \cite{adolf_physiology_1947,king_brief_1878}. While more modern research has confirmed cognitive-motor deficits under adverse physiological stress (e.g., sleep deprivation, HYPO) using more sophisticated instrumentation \cite{lieberman_severe_2005}, the more pertinent question remains understanding the potential impairments in cognitive-motor performance following HYPO magnitudes likely to be encountered during normal military/occupational/athletic scenarios \cite{sawka_american_2007}. Four early studies were seemingly the first to examine the specific effects of HYPO on cognitive performance, observing mixed results \cite{bijlani_effect_1980,sharma_differential_1983,sharma_influence_1986,gopinathan_role_1988}.  Three of these studies suggested cognitive-motor functions of mental arithmetic, executive function, and information processing were impaired at moderate levels of HYPO (-2\% BM), with some suggestions that an increased magnitude of HYPO (-4\% BM) resulted in further cognitive-motor deterioration \cite{gopinathan_role_1988} in hot conditions \cite{sharma_differential_1983,sharma_influence_1986}.    

Following these early studies, various levels of HYPO, cognitive domains, and methods to induce HYPO have been utilized across a number of investigations, further complicating the interpretation of previous literature. While most review articles variability in study design largely explains the disparate literature \cite{masento_effects_2014,grandjean_dehydration_2007}, studies controlling for some of these variables have not always supported this claim. For example, when HYPO magnitude was controlled (2.8\%), HYPO induced by both exercise heat stress and heat stress alone deteriorated short term memory and perceptive discrimination \cite{cian_influence_2000}. This study was followed up a year later \cite{cian_effects_2001} with similar findings with additional perceptual measures which suggested HYPO elevated fatigue during the cognitive-motor battery. How other HYPO methods (e.g., fluid restriction vs exercise-heat stress) compare and potentially impact cognitive-motor responses following body water deficits merits further investigation.

Cognitive-motor task/domain selection may be a larger contributor to the wide variation within the previous literature. Most previous studies employed a multi-modal cognitive-motor battery assessing a wide array of cognitive-motor domains (e.g., executive function, working memory, reaction time, information processing). As a result, there are commonly observed within-study discrepancies, with some aspects of cognitive-motor function being impaired while others remain unaffected \cite{ganio_mild_2011}. While these results may suggest some cognitive-motor domains are more affected by water deficits than others, no concrete mechanism has elucidated why HYPO may impair cognitive-motor performance. One commonly put forth suggestion is that HYPO may impair the higher-order cognitive-motor functions (e.g., executive functions). Evidence for this has been observed previously \cite{tomporowski_effects_2007}, with the more difficult variation of an executive function showing greater errors than the less complex variation. However, across the literature, this idea has not always been supported. Executive functions (e.g., working memory, executive function, inhibitory control) have not been uniformly impaired across all studies \cite{wittbrodt_exercise-induced_2015,armstrong_mild_2012,danci_voluntary_2009,kakos_improving_2013,turner_mild_2017}, however, it should be noted that some of these previous studies elicited modest levels of HYPO ($<$2\% BM loss). Furthermore, some have observed decrements in lower level tasks, such as simple reaction time \cite{barroso_hydration_2014,mcmorris_heat_2006}, indicating cognitive-motor performance impairments are likely not isolate to ``complex'' tasks. How each cognitive-motor domain is affected by HYPO requires further investigation.    

Recently, some investigations have attempted to better understand the potential mechanisms of how HYPO may impair cognitive-motor function by employing neuroimaging techniques. The first such investigation \cite{szinnai_effect_2005} utilized electroencephelography (EEG) and assessed auditory event-related potentials (P300) with an oddball paradigm. Measuring three central electrodes on the frontal, central, and parietal regions, there was no impact of HYPO (2.6\% BM loss) on either latency or amplitude of the auditory event-related potentials \cite{szinnai_effect_2005}. However, despite no changes in EEG responses, subjects did report increased levels of effort and concentration required to complete the cognitive-motor tasks following HYPO, suggesting the tasks were more strenuous although not to the extent of impairing function or performance \cite{szinnai_effect_2005}. 

Following this initial study, others have employed functional magnetic resonance imaging (fMRI) to further examine the effects of HYPO on cognitive-motor performance and function. fMRI has the benefit of added spatial resolution along with the ability to investigate activations within cortical and subcortical brain areas \cite{mehta_neuroergonomics:_2013}, which may allow better understanding of a potential mechanism by which HYPO may impair cognitive-motor performance. One study has investigated blood oxygen level dependent (BOLD) responses to a cognitive-motor task following HYPO \cite{kempton_dehydration_2011}. During an executive-function task (Tower of London), HYPO (1.6\% BM loss) elicited increased BOLD responses within fronto-parietal regions \cite{kempton_dehydration_2011}. However, HYPO did not alter executive-function task performance \cite{kempton_dehydration_2011}, and therefore the authors suggested elevated BOLD following HYPO indicate inefficient neural processing which might not be reflected in behavioral impairments. More recently, another EEG study \cite{watson_mild_2015} suggested alpha (8-11 Hz) and theta (4-7 Hz) power within the motor brain regions (C3, C4 electrodes) may be increased following marginal HYPO (1.1\% BM loss) during a prolonged, monotonous, driving task. This was coupled with an increase in errors following HYPO, suggesting that, over an extended (120 min) period, HYPO may increase the neural resource requirement leading to impaired performance. However, given the analysis within the previous study, it is difficult to elucidate whether this increase in brain activation is related to increased effort, task-related impairments, or another mechanism. Lastly, one study has examined the effects of HYPO on pain-evoked-activations (cold presser test) with fMRI \cite{ogino_dehydration_2014}. HYPO elicited greater BOLD responses within the anterior cingulate, insula, and thalamus compared to controlled conditions along with a decreased pain threshold, suggesting somatosensory function may also be altered. Taken together, these initial neuroimaging studies have begun to provide unique insight into how HYPO alters neural resource requirements/activation patterns during cognitive-motor or sensory tasks which help understand a mechanism by which body water deficits affects cognitive-motor function and performance.

\section{Hypohydration and Visuomotor Function}
Within studies assessing cognitive-motor responses to HYPO, cognitive-motor performance is typically assessed with a computerized task requiring subjects to produce a movement (e.g., button press) and therefore activating the visuomotor system. While the primary and supplementary motor cortices were once believed to govern motor movement, more recent evidence suggests these areas also respond to the presentation of visual stimuli within the frontal eye field \cite{fadiga_visuomotor_2000}. The entire visuomotor system consists of a multitude of fronto-parietal brain areas (prefontal cortex, supplementary eye field, frontal eye field, primary motor cortex), subcortical structures (thalamus, basal ganglia, hippocampus), and the cerebellum which combine to encode a motor program to facilitate completion of an action (e.g., pressing of button) \cite{murray_role_2000}. This thesis will further explore visuomotor timing during simple, externally paced finger tapping tasks. Within these visually paced tasks (e.g., responding to a blinking light), researchers have primarily been interested in the timing mechanisms and potential influences \cite{ruspantini_considerations_2011}. The tasks can take many variations including the number of fingers required, nature of stimulus (visual or auditory), and whether the pacing is external (e.g., stimulus on screen) or internal (e.g., tapping at an internally-driven rhythm) \cite{witt_functional_2008}. Furthermore, the performance effects can also widely vary based on the frequency of tapping \cite{bove_effects_2007}.

Broadly, the production of a motor response occurs in two phases: motor planning and motor execution \cite{crammond_prior_2000}. Motor planning consists of the cognitive steps involved before movement occurs: observing the environment (i.e., stimulus identification), integrating task rules (i.e., if stimulus is red, press right button), determining the motor goal (i.e., press right button), and the abstract kinematics (i.e., how to move right) \cite{wong_motor_2015}. It is believed motor planning, because of the information processing involved, dictates the measured reaction time within visuomotor studies \cite{wong_motor_2015}. Although motor planning and execution are presented here as occuring in serially, some have suggested neurons within the primary motor cortex and dorsal premotor cortex may contribute simultaneous to both motor planning and motor execution \cite{crammond_prior_2000}. Motor execution is the initiation of a movement to accomplish the motor goal \cite{kaufman_roles_2013}. Motor responses occur via integration of the motor thalamus, basal ganglia (globus pallidus internus, substantia nigra pars reticulata), cerebellum, and the associative, premotor, and motor areas of the cerebral cortex (Figure \ref{fig:motor_thalamus} \cite{rizzolatti_organization_1998,bosch-bouju_motor_2013}. Within motor execution, it appears that the motor thalamus is integral to integrating various pieces of information (e.g., motor program, temporal pattern, motivation, action selection, action selection) to select/execute the correct motor movement \cite{bosch-bouju_motor_2013}. However, the thalamus has a large well establish sensory role with integrations from the visual cortex along with mediating arousal, attention, and other sensory stimuli \cite{nakajima_thalamic_2017}. Because HYPO elicits well-defined deleterious perceptual changes \cite{engell_thirst_1987,armstrong_mild_2012,lieberman_severe_2005}, the thalamus, and thus visuomotor function, could be uniquely challenged by body water deficits. 

\begin{figure}
	\centering
	\includegraphics[width=10cm]{figures/"motor_thalamus".pdf}
	\caption{Connections within the thalamus during motor execution. SNPi = substantia nigra pars reticularis, GPi = globus pallidus internus. Thalamusic nuclei: ventral anterior (VA), ventral later anterior (VLa), ventral lateral posterior (VLp), ventral media (VM). \textit{Adapted from \cite{bosch-bouju_motor_2013}.}}
	\label{fig:motor_thalamus}
\end{figure}

Despite this potential risk for visuomotor impairments by thalamus overload (or other neurophysiological mechanism), previous research has not systematically investigated the effects of HYPO on motor planning and motor execution. Furthermore, there is no neuroimaging evidence, to our knowledge, examinging how the brain areas integral to visuomotor function (motor cortex, supplementary motor cortex, basal ganglia, thalamus, cerebellum, parietal cortex) are impacted by HYPO. Previous research has examined some components of visuomotor function (Figure \ref{tbl:m_c_studies}), however, the results are not clear. One example of this discrepancy in the literature occurred in two studies from the same lab \cite{cian_effects_2001,cian_influence_2000}. Using an identical HYPO protocol, magnitude of HYPO, and unstable tracking task, one study reported impaired motor coordination \cite{cian_influence_2000} while the subsequent study did not \cite{cian_effects_2001}. Despite the equivocal literature, reports of slowed visuomotor reaction time \cite{wong_effects_2014} and decreased motor coordination accuracy \cite{sharma_influence_1986,cian_influence_2000} have been observed following HYPO, suggesting both motor planning and motor execution may be impaired. Changes within visuomotor task performance may manifest/contribute to deteriorations in skilled task performance, which has been previously observed \cite{baker_progressive_2007,smith_effect_2012,nuccio_fluid_2017}.

%%% MOTOR FUNC STUDIES
\begin{table}
	\caption{Studies utilizing aspect of cognitive-motor functioning or occupation specific task following hypohydration (HYPO). ND = No difference.}
	\centering
	\begin{tabular}{llll} 
		\hline
		\footnotesize\textbf{Study} & \footnotesize\textbf{\% BM Loss} & \footnotesize\textbf{Task} & \footnotesize\textbf{Self-Reported Findings} \\
		\hline
		\footnotesize Sharma et al. \cite{sharma_influence_1986} & \footnotesize 1-3 & \footnotesize Psychomotor Stylus Test & \footnotesize Impaired at ${\ge}$2\% BM loss \\
		\footnotesize Patel et al. \cite{patel_neuropsychological_2007} & \footnotesize 2.5 & \footnotesize Balance Error Scoring System & \textemdash \\
		\footnotesize Cian et al. \cite{cian_influence_2000} & \footnotesize 2.8 & \footnotesize Unstable Tracking Task & \footnotesize Increased Deviation (p ${<}$ 0.05) \\
		\footnotesize Cian et al. \cite{cian_effects_2001} & \footnotesize 2.8 & \footnotesize Unstable Tracking Task & \textemdash \\
		\footnotesize Wong et al. \cite{wong_effects_2014} & \footnotesize 1.4 - 2 & \footnotesize Psychomotor Reaction Time Test & \footnotesize Decreased Speed (p ${<}$ 0.05) \\
		\hline		    
	\end{tabular}
	\label{tbl:m_c_studies}
\end{table}


\section{Research Aims}
Hypohydration is known to impair performance in athletic / occupational / military environments \cite{iom_dietary_2004}, however, the effects on cognitive-motor performance are much less understood. Cognitive-motor performance is typically assessed with a computerized task where subjects produce a movement (e.g., button press), requiring activation of the areas involved in creating a motor response (i.e., visuomotor system). However, to date, the effect of HYPO on visuomotor performance is unclear. Furthermore, relatively few studies have integrated any neuroimaging measures during the performance of cognitive tests following HYPO (and none during visuomotor tasks) \cite{kempton_dehydration_2011,watson_mild_2015,szinnai_effect_2005}, providing minimal evidence regarding body water deficits and resulting brain function. By testing the following specific aims, we evaluated how HYPO impacts brain structure along with performance and function in visuomotor tasks emphasizing the two phases of visuomotor function: motor planning and motor execution. 

%% Aim 1
In Chapter 2, we address research \textbf{Aim 1: To utilize a systematic review and meta-analysis to quantitatively determine whether previous literature suggests hypohydration affects cognitive-motor performance}. Based on conclusions from multiple review articles \cite{masento_effects_2014,lieberman_hydration_2007,grandjean_dehydration_2007}, it is still unclear: i) whether hypohydration impairs cognitive-motor performance and ii) what the mechanism relating to these potential impairments. To assess this, a systematic review will be completed along with pertinent review articles scanned to collect all studies examining the effects of HYPO on cognitive-motor performance. All cognitive-motor tasks will be classified into specific domains (e.g., working memory, reaction time, attention) based upon a previously published categorization protocol \cite{chang_effects_2012}. Further study information will be captured and analyzed such as the method to elicit hypohydration, level of hypohydration achieved, subject demographics, subject fitness level, and duration from end of hypohydration intervention to cognitive-motor testing (during exercise-based hypohydration interventions). 

We hypothesize hypohydration will:
\begin{itemize}
	\item \textit{Have a small, but significant, negative affect on cognitive-motor performance}
	\item \textit{Effect executive functions to a greater extent than other cognitive-motor domains}
	\item \textit{Have a negative association between body mass loss and magnitude of cognitive-motor impairment}
\end{itemize}
	
% Aim 2
In Chapter 3, we address \textbf{Research Aim 2: To examine the effect of exercise-heat stress with and without fluid replacement on brain structure, function, and visuomotor performance}. While the literature presents disparate findings (Table \ref{tbl:mri_deh}), previous studies have not: i) established baseline hydration status, ii) included a matched exercise-heat stress control trial, and iii) elicited a HYPO magnitude sufficient to markedly alter physiological status (${\ge}$2\% BM loss) \cite{cheuvront_biological_2010}. Furthermore, no study has systematically assessed the effects of HYPO on the visuomotor system, which governs many of the human-system interactions \cite{buhusi_what_2005} and therefore is directly applicable to many military and occupational scenarios. Thus, a comprehensive study was needed to investigate the effects of hypohydration (${\sim}$3\% BM loss) on brain structure, function, and visuomotor performance. 

To examine this question, we instituted a protocol consisting of three experimental trials: a resting control and two exercise-heat stress bouts matched for duration and intensity, one with and one without fluid replacement. Following exercise-heat stress, anatomical scans were completed along with fRMI scans during completion of a visuomotor task. To assess visuomotor performance, we employed a visually paced unimanual finger tapping task (visuomotor pacing task, VMPT) requiring repeated button presses at regular and irregular intervals. Performance and brain function (via fMRI BOLD responses) during the VMPT will be examined to explore possible hypohydration-mediated effects on the visuomotor system. Anatomical scans will be analyzed to determine the effects of hypohydration on individual brain structures, which has not been previously investigated.

We hypothesized hypohydration (${\sim}$3\% BM loss) elicited by exercise-heat stress will:
	\begin{itemize}
		\item \textit{Not alter total brain volume}
		\item \textit{Induce expansion of the ventricular system}
		\item \textit{Decrease periventricular structure volume (i.e., those surrounding the ventricles)} 
		\item \textit{Increase BOLD responses during visual paced finger tapping while not impairing task performance}
		\item \textit{Impair performance over time during visual paced finger tapping}
	\end{itemize}

%% Aim 3
In Chapter 4, we address \textbf{Research Aim 3: To evaluate which phases of visuomotor processing (e.g., motor planning, motor execution) are impaired due to moderate hypohydration}. The VMPT was designed to minimize motor planning requirements, as the stimulus-response combinations were intentionally limited to one. To investigate motor planning, a paced choice reaction task (PCRT) will be implemented. The PCRT, like the VMPT, is visually paced but requires bimanual input and features weighted response frequencies between the left and right index fingers. The primary goal of the PCRT is to manipulate the amount of prior information which can potentially influence motor planning time \cite{wong_motor_2015}. To further examine the effect of hypohydration on motor planning demands, brain activations via EEG will be measured during PCRT completion to examine neural responses throughout the task. EEG has superior temporal resolution to fMRI \cite{mehta_neuroergonomics:_2013} and therefore may help elucidate how hypohydration influences vsiuomotor timing. Secondly, indices of mental workload (measured via the NASA-TLX scale) during the PCRT will help better understand the effect of hypohydration on perceived mental workload.

We hypothesized moderate hypohydration (${\sim}$3\% BM loss) elicited by exercise-heat stress will:
        
\begin{itemize}
	\item \textit{Increase reaction time difference between dominant and non-dominant weights compared to control conditions with the difference exacerbated difference with increased weighted difference between dominant and non-dominant side}
	
	\item \textit{Result in greater task-specific area activations within the pre-motor area cortex and supplementary motor areas, indicating greater neural resources required during motor planning.}
	
	\item \textit{Result in greater task-specific mental workload, as measured by the NASA-TLX scale.}
\end{itemize}
