\chapter{Introduction and Background}

There are reasons to believe body water deficits (hypohydration; HYPO) can have adversely impact the central nervous system (CNS). Severe levels of HYPO (e.g., water restriction in the desert) result in CNS dysfunction such as decreased mental alertness, personality changes (e.g., hostility), and headaches \cite{adolf_physiology_1947, king_brief_1878}. However, at more modest states of HYPO, the impact is less clear. It is suggested that certain populations (e.g., elderly) may be at greater risk for confusion and delirium, leading to increased frequency, duration, and severity of hospitalizations due to HYPO \cite{inouye_delirium_2006,warren_burden_1994}. Small to moderate levels of HYPO may also degrade human-system performance reliant on adequate CNS function, as evidenced by impaired driving ability \cite{watson_mild_2015} and performance during occupation-specific simulations (e.g., pilot proficiency) \cite{lindseth_effects_2013}. Indirect evidence also suggests occupational accident rates are increased during hot weather/summer months when HYPO is likely \cite{kenefick_hydration_2007, kenefick_dehydration_2012}. Lastly, some have suggested that the acute act of drinking water (when marginal body water deficits of ${<}$ 1\% body mass (BM) loss) can improve memory in both school children and adults \cite{benton_effect_2009,benton_small_2015,benton_minor_2016,fadda_effects_2012}. However, despite these numerous observations, the published peer-reviewed literature remains inconsistent regarding which CNS functions are most susceptible to HYPO and what magnitude of HYPO can be tolerated before decrements in cognitive task performance occur.

\section{Physiological Responses to Hypohydration}
Hypohydration elicits severe consequences to physiological homeostasis \cite{iom_dietary_2004}. In the body, water is stored within three compartments: plasma (8\% of total body water), interstitial fluid (25\%), and intracellular fluid (67\%) (Figure \ref{fig:fluidVolumes}) \cite{sawka_thermoregulatory_1985}, with water transfer among these compartments largely governed by Starling forces (capillary and/or osmotic exchange). The composition of fluid lost (e.g., sweating or diarrhea) influences the osmotic gradient and, therefore, dictates physiological responses to HYPO. Because these physiological responses vary in their sensitivity (i.e., level of body water loss which elicits up-regulation of response mechanism), HYPO is classified as either intracellular or extracellular \cite{cheuvront_dehydration:_2014}. For the purpose of this thesis, only intracellular HYPO will be considered. Intracellular HYPO occurs via water losses through non-illness (e.g., diarrhea) or non-medicinal (e.g., diuretic) means such as exercise, heat stress, or fluid restriction. 

\begin{figure}
	\centering
	\includegraphics[width=8cm]{figures/"Fluid Volumes".pdf}
	\caption{Distribution of body water among compartments. \textit{Adapted from \cite{sawka_thermoregulatory_1985}}.}
	\label{fig:fluidVolumes}
\end{figure}

As previously stated, HYOP elicited by exercise-heat stress (EHS) induce intracellular hypohydration. During EHS, elevations in sweat rate (a byproduct of increased metabolic heat production from the exercising muscles) \cite{sawka_integrated_2011} results in marked losses of sweat, which is hypotonic fluid to blood plasma (i.e., less solute for given volume). Therefore, more relative water is lost compared to solute, which results in an osmotic gradient between the blood plasma and interstitial fluid. The blood plasma becomes hypertonic, which, via Starling forces, draws water from the interstitial compartment into the blood plasma and from the intracellular to interstitial compartment \cite{cheuvront_dehydration:_2014}. Intracellular HYPO is also termed hypertonic hypovolemia, as a loss of blood volume (hypovolemia) is coupled with a hypertonic blood plasma \cite{darrow_changes_1935}.  

\section{Cellular Mechanisms of Central Water Homeostasis}
Although osmoregulation occurs via activation of both peripheral and central osmoreceptors, this thesis will focus primarily central mechanisms of water homeostasis. The primary central osmoreceptors are located in brain areas without presence of a blood-brain barrier, the circumventricular organs, specifically the \textit{organum vasculosum laminae terminalis} (OVLT), which sits at the base of the third ventricle \cite{bourque_central_2008}. As the OVLT is exposed to hypertonic fluid, the cellular volume shrinks (intracellular HYPO), leading to a depolarized state and increased action potential encoding \cite{bourque_central_2008}. Secondly, although not strictly osmoreceptors, many cellular types show osmoreceptive properties. Glial cells, specifically astrocytes, aid in regulating brain water and ion homeostasis in addition to maintaining the blood brain barrier are involved in the maintenance of the blood–brain barrier \cite{simard_neurobiology_2004}. Following the sensing of hypertonicity within the blood, arginine vasopressin (AVP) is released both systemically (to the body) and centrally (to the brain) \cite{simard_neurobiology_2004}. AVP therefore acts upon astrocytes to accomplish two things: i) influx of Na${^+}$, K${^+}$, and Cl${^-}$, ii) over-expression of aquaporin 4 to allow influx of water \cite{simard_neurobiology_2004}.   

These osmoprotective mechanisms are particularly robust within the brain, which maintains total volume despite large body water losses \cite{nose_distribution_1983,cipolla_cerebral_2009}. This likely results from neuroprotective mechanisms to maintain water content in brain tissue during osmotic stress such as ion pumps and leak channels \cite{cipolla_cerebral_2009,cserr_extracellular_1991}. The brain, unlike other body tissues, does not allow water influx by Starling forces, as the presence of high electrical resistance tight junctions (${\sim}$100x greater than capillaries) within the blood brain barrier (BBB) prevent movement of electrolytes and other hydrophilic substances \cite{kimelberg_water_2004,cipolla_cerebral_2009,ropper_hyperosmolar_2012,bain_cerebral_2015}. These protective mechanisms are also bolstered by astrocytes, which assist in maintaining structural integrity of the tight junctions, regulating ionic homeostasis, and buffering K${^+}$ \cite{cipolla_cerebral_2009}. Secondly, the brain produces organic osmolytes (myo-inositol, betaine, taurine, sorbitol) which help preserve cellular volume by increasing ionic concentration within the cell, equilibrating the osmotic gradient with the hypertonic exterior \cite{de_petris_cell_2001,weed_pressure_1919}. During chronic hypertonicity (extracellular osmolality greater than intracellular osmolality), organic osmolytes accounted for 35\% of brain tonicity (with no change in brain water content) \cite{lien_effects_1990} and were up-regulated by 44\% in an infant suffering severe HYPO \cite{lee_organic_1994}. However, because organic osmolyte gene encoding occurs following sensing of hypertonicity, appearance within the brain tissue is delayed \cite{de_petris_cell_2001, gullans_control_1993}. As a result, osmolyte-mediated equilibration of neuronal water content does not occur immediately following acute exposure to hypertonic environment \cite{ayus_effects_1996}. Therefore, although humans may be protected against large brain water perturbations, the delayed time course of osmolyte production may result in vulnerabilities to brain water homeostasis during acute HYPO as elicited by exercise-heat stress. 

\section{Neuroimaging and Hypohydration}
Until recently, the ability to image and analyze an \textit{in vivo} human brain has not been available. As a result, the first usage of this technology to scan the human brain following HYPO occurred in 2005 \cite{dickson_effects_2005, duning_dehydration_2005}. However, these two studies found different conclusions, with one observing finding no differences in brain volume following HYPO \cite{dickson_effects_2005}, but the other finding a significant decrease in total brain volume \cite{duning_dehydration_2005} (Table \ref{tbl:mri_deh}). Following these initial studies, others have also investigated brain responses elicited by body water losses, although the literature remains equivocal and present difficulty definitive understanding of how the human brain responds to HYPO. 

A total of eight studies have investigated the effects of HYPO on brain morphology (Table \ref{tbl:mri_deh}), with most reporting changes in total brain volume and/or lateral ventricle size. Only one study has observed decreases in total brain volume following HYPO, with a 16h fluid restriction protocol (1.6 \% body mass (BM) loss) decreasing brain volume by ${\sim}$0.5\%, although it should be noted individual responses ranged from +0.2 to -2.4\% \cite{duning_dehydration_2005}. The authors attributed decreased brain volume to the osmotic gradient induced by hypertonic hypovolemia shrinking the astrocytes which compromised fluid movement within the brain \cite{duning_dehydration_2005}. However, every other study has observed no changes in total brain volume with HYPO ranging from 0.8 - 2.9 \% body mass (BM) loss \cite{watson_effect_2010,kempton_effects_2009,kempton_dehydration_2011,meyers_does_2016}, suggesting this finding may be aberrant with regards to the literature at large.

Secondly, given the association between fluid balance and fluid-containing centers of the brain, many studies have attempted to determine changes within the lateral ventricles following HYPO. However, unlike total brain volume, the data on this topic is much less clear. While some studies have observed no significant difference \cite{meyers_does_2016, dickson_effects_2005, streitburger_investigating_2012,meyers_does_2016,dickson_effects_2005}, others have reported either an increase \cite{kempton_dehydration_2011,kempton_effects_2009} or decrease \cite{watson_effect_2010} in lateral ventricular volume following HYPO. Furthermore, some have suggested that lateral ventricle volume expansion significantly correlates to BM loss \cite{kempton_dehydration_2011, dickson_effects_2005}, suggesting magnitude of body water loss could elicit a graded increase in lateral ventricular volume. However, in both instances, these correlations were observed over a small range of HYPO (${\sim}$1\%), individuals with both ventricular expansion and shrinkage within the sample, and there was wide dispersion of ventricular volume changes (${\ge}$ 14\%). 

A common belief has been that ventricular expansion occurs from shrinking astrocytes limiting fluid movement throughout the brain \cite{kempton_dehydration_2011}. However, how other brain structures besides the ventricles change following HYPO has rarely been studied. Streitburger et al. \cite{streitburger_investigating_2012} found decreases in both white and grey matter in areas with close proximity to the brain ventricles, however, this was when hyperhydration was compared to HYPO. Although these are intriguing findings, the small sample size (n = 6) and minimal level of HYPO (-2.3\%) are limiting factors. Furthermore, the voxel-based morphology approach did not allow for the analysis of individual brain structures, and therefore any relationship to changes in brain structures and changes in brain function cannot be established. Therefore, this thesis attempted to increase both sample size, magnitude of HYPO, and the parcellation of individual brain areas to better investigate the effects of HYPO on brain structures. 

%%%%%%%% Imaging Table
\begin{table}
\caption{Previous Studies Using MRI to Investigate Hypohydration (HYPO) on Brain Morphology. EHS = Exercise Heat Stress, ND = No Difference (from control), CSF = cerebrospinal fluid, FR = Fluid Restriction. * Different versus a hyperhydration condition.}
\begin{center}

\begin{tabular}{lcccc} 
\hline
\footnotesize\textbf{Citation} & \footnotesize\textbf{Subject n} & \footnotesize\textbf{Method of HYPO \& Control} & \textbf{${\Delta}$ BM} & \footnotesize\textbf{Findings} \\
\hline

\footnotesize Meyers et al. \cite{meyers_does_2016} & \footnotesize 20 \scriptsize(11 M) & \footnotesize Fluid Restriction & \footnotesize -0.8\% & \footnotesize ND \\

\footnotesize Duning et al. \cite{duning_dehydration_2005} & \footnotesize 10 \scriptsize(3 M) & \footnotesize FR; Pre-FR & \footnotesize -1.6\% & \footnotesize ${\downarrow}$ Brain Volume \\

\footnotesize Kempton et al. \cite{kempton_dehydration_2011} & \footnotesize 10 \scriptsize(5 M) & \footnotesize EHS; EHS with fluid & \footnotesize -1.6\% & \footnotesize ${\uparrow}$ \footnotesize Lateral Ventricle \\

\footnotesize Kempton et al. \cite{kempton_effects_2009} & \footnotesize 7 \scriptsize(7 M) & \footnotesize EHS; Pre-EHS & \footnotesize -2.2\% & \footnotesize ${\uparrow}$ \footnotesize Lateral Ventricle \\

\footnotesize Dickson et al. \cite{dickson_effects_2005} & \footnotesize 6 \scriptsize(6 M) & \footnotesize EHS; Pre-EHS & \footnotesize -2.3\% & \footnotesize ND \\

\footnotesize Streitburger et al. \cite{streitburger_investigating_2012} & \footnotesize 6 \scriptsize(3 M) & \footnotesize FR; Pre-FR & \footnotesize -2.3\% & \footnotesize ${\downarrow}$ Grey Matter*, ${\downarrow}$ White Matter* \\

\footnotesize Watson et al. \cite{watson_effect_2010} & \footnotesize 8 \scriptsize(8 M) & \footnotesize EHS; Pre-EHS & \footnotesize -2.9\% & \footnotesize ${\downarrow}$ Total Ventricle \& CSF \\

\footnotesize Nakumura et al. \cite{nakamura_correlation_2014} & \footnotesize 14 \scriptsize(12 M) & \footnotesize FR; Pre-FR \scriptsize(2wk prior) & \footnotesize ? & \footnotesize ND \\	
\hline

\end{tabular}
\label{tbl:mri_deh}
\end{center}
\end{table}



\section{Research Aims}
Hypohydration is known to impair performance in athletic / occupational / military environments \cite{iom_dietary_2004}, however, the effects on cognitive-motor performance are much less understood. Cognitive-motor performance is typically assessed with a computerized task where subjects produce a movement (e.g., button press), requiring activation of the areas involved in creating a motor response (i.e., neuromotor system). However, to date, the effect of HYPO on cognitive-motor performance is unclear given the disparate findings throughout the literature. Furthermore, relatively few studies have integrated any neuroimaging measures during the performance of cognitive tests following HYPO (and none with cognitive-motor functioning) \cite{kempton_dehydration_2011,watson_mild_2015,szinnai_effect_2005}, providing minimal evidence regarding body water deficits and resulting brain function. By testing the following specific aims, we evaluated how HYPO impacts brain structure along with performance and function in cognitive-motor tasks emphasizing the two phases of cognitive-motor function: motor planning and motor execution \cite{crammond_prior_2000}. 

%% Aim 1
We hypothesize that:

\begin{itemize}
	\item Hypohydration will induce a small negative effect on cognitive performance
	\item Magnitude of hypohydration will be significantly related to magnitude of cognitive performance impairment 
	\item There will be a significant difference in those studies hit a body mass loss ${>}$2\% compared to those eliciting under 2\% body mass loss  
\end{itemize}

In Chapter 2, we address research \textbf{Aim 1: To quantify changes in brain structures (ventricular system, cerebellum, subcortical areas) due to moderate hypohydration elicited by sweat loss during exercise-heat stress}. While the literature presents disparate findings (Table \ref{tbl:mri_deh}), most studies i) do not control for baseline hydration status, ii) do not integrate a matched exercise-heat stress control trial, and iii) have not uniformly elicited a magnitude of HYPO which results in a marked change in hydration status (${\ge}$2\% BM loss). Therefore, we instituted a protocol with two controls: a resting control and an exercise-heat stress bout matched for duration and intensity with fluid loss being replaced with water. Following exercise-heat stress, we also allowed for a recovery bout (${\sim}$ 40min) before brain imaging. Secondly, no previous study examining the effect of HYPO on brain structure has examined changes to individual brain structures, which has previously precluded the inference of any structure-cognitive function/performance relationships.

We hypothesized hypohydration (${\sim}$ 3\% BM loss) elicited by exercise-heat stress will:
	\begin{itemize}
		\item \textit{Not alter total brain volume}
		\item \textit{Induce expansion of the ventricular system}
		\item \textit{Decrease intracellular volume of periventricular structures (i.e., those surrounding the ventricles)} 
	\end{itemize}

%% Aim 2
In Chapter 3, we address research \textbf{Aim 2: To examine the association between structural changes, neural resource requirements, and/or performance errors during a rhythmically paced fine motor task.} Using the same experimental design as Aim 1, we wanted to better understanding potential impairments to the neuromotor system following HYPO. We instituted a visually paced unimanual finger tapping task (motor pacing task, MPT) which requires repeated button presses at regular or irregular intervals. Given the minimal decision-making elements, the MPT emphasized motor execution (initiation of a movement to accomplish the motor goal) \cite{kaufman_roles_2013}.To better understand the functional implications of cognitive-motor performance, we also instituted functional magnetic resonance imaging (fMRI) to investigate the neural resource requirements following HYPO. Previous research has identified that neural activity may increase \cite{kempton_dehydration_2011,watson_mild_2015} following HYPO, and therefore functional issues may underpin the performance responses. 

We hypothesized moderate HYPO elicited by exercise-heat stress will:

\begin{itemize}
	\item \textit{Increase BOLD responses required for motor execution of visual paced finger tapping in order to maintain task performance}
	\item \textit{Not degrade overall MPT performance (increase reaction time or reduce temporal accuracy) during both the regular and irregular timed task variations}
	\item \textit{Exacerbate performance degradation over time during MPT completion (i.e., lower temporal accuracy during the last five minutes compared to both control conditions)}
\end{itemize}

%% Aim 3
In Chapter 4, we address research \textbf{To evaluate which phases of neuromotor processing (e.g., motor planning, motor execution) are impaired due to moderate HYPO}. The MPT was designed to have a minimal motor planning phase, as only one stimulus-response combination was possible. To investigate the other broad phases of cognitive-motor function, motor planning, a paced choice reaction task (PCRT) was implemented. The PCRT, like the MPT, was visually paced but is bimanual and featured weighted response frequencies between fingers. The goal was to manipulate the cognitive steps involved before movement occurs: observing the environment (i.e., stimulus identification), integrating task rules (i.e., if stimulus is red, press right button), determining the motor goal (i.e., press right button), and the abstract kinematics (i.e., how to move right) \cite{wong_motor_2015}. Secondly, we instituted electroencephalography (EEG) during the PCRT to examine neural changes throughout the task. EEG has superior temporal resolution to fMRI \cite{mehta_neuroergonomics:_2013}. Secondly, indices of core affect and mental workload during the PCRT will help better describe perceptual responses during the task.

We hypothesized moderate HYPO elicited by exercise-heat stress will:
        
\begin{itemize}
	\item \textit{HYPO will have an greater difference between dominant and non-dominant side reaction times compared to the control conditions and this difference will be exacerbated as the weighting between dominant and non-dominant side increases.}
	
	\item \textit{Result in greater task-specific area activations within the pre-motor area cortex and supplementary motor areas, indicating greater neural resources required during motor planning.}
	
	\item \textit{Result in greater task-specific mental workload, as measured by the NASA-TLX scale.}
\end{itemize}


