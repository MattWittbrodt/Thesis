\chapter{Aim 2}

\subsection{Introduction}


 Motor responses occur via integration of the basal ganglia (motor thalamus, globus pallidus internus, substantia nigra pars reticulata), cerebellum, and the associative, premotor, and motor areas of the cerebral cortex \cite{rizzolatti_organization_1998,bosch-bouju_motor_2013}. Despite the importance of these phases to cognitive task performance, previous research has not systematically investigated the effects of HYPO on motor planning and motor execution. Tasks featuring a large neuromotor component have identified conflicting results.
 
 
 
 
If body water losses exceed the reference change volumes, there appears to be marked decrements in exercise and functional performance. Early evidence from Adolph and colleagues \cite{adolf_physiology_1947} observed soldiers achieving less marching distance with less water provisions. Furthermore, Adolph also observed changes in affect of these individuals with HYPO, including worsened metal state (i.e., irritability), inability to navigate, and decreased morale \cite{adolf_physiology_1947}. More recently, many studies have been conducted with laboratory settings to examine the effect of HYPO on human performance. A recent review has identified impaired endurance exercise performance (range of impairment: -6 to -54\%) with HYPO ${\ge}$2\% BM loss, although the effects on muscular strength are less convincing \cite{sawka_hypohydration_2015,savoie_effect_2015}. Evidence also suggests that impaired exercise performance may occur without alterations to the local metabolic environment within the exercise muscle, as pH and Pi-to-ATP ratio was unchanged following submaximal knee extension exercise time to performance test \cite{montain_hypohydration_1998}. Lastly, skilled movements may also be impacted, as some have observed impairments in skilled basketball tasks \cite{baker_progressive_2007} and golf performance \cite{smith_effect_2012} following HYPO. 

\subsection{Functional}
 Cognitive function is typically assessed with a computerized task requiring subjects to produce a movement (e.g., button press) which requires activation of the areas involved in creating a motor response (i.e., neuromotor system). Broadly, the production of a motor response occurs in two phases: motor planning and motor execution \cite{crammond_prior_2000}. Motor planning consists of the cognitive steps involved before movement occurs: observing the environment (i.e., stimulus identification), integrating task rules (i.e., if stimulus is red, press right button), determining the motor goal (i.e., press right button), and the abstract kinematics (i.e., how to move right) \cite{wong_motor_2015}. Subsequently, motor execution is the initiation of a movement to accomplish the motor goal \cite{kaufman_roles_2013}. Motor responses occur via integration of the basal ganglia (motor thalamus, globus pallidus internus, substantia nigra pars reticulata), cerebellum, and the associative, premotor, and motor areas of the cerebral cortex \cite{rizzolatti_organization_1998,bosch-bouju_motor_2013}. Despite the importance of these phases to cognitive task performance, previous research has not systematically investigated the effects of HYPO on motor planning and motor execution. Tasks featuring a large neuromotor component have identified conflicting results. Fine motor speed may either improve \cite{bandelow_effects_2010} or be unaffected \cite{hogervorst_cognitive_1996} following HYPO, but both motor coordination \cite{sharma_influence_1986,cian_influence_2000} and skilled task performance \cite{baker_dehydration_2007, watson_mild_2015,lieberman_severe_2005} may deteriorate. 

\subsection{Thirst}
Thirst, in response to body water deficits, was one of the first mammalian homeostatic mechanisms every discovered. Indeed, the sensation of thirst is a prominent sensations triggered in response to body water deficits \cite{bourque_central_2008}, with the severity of symptoms positively associated with severity of body water losses \cite{engell_thirst_1987}. Although the hypothalamus was believed to be the classical thirst center, a more integrated neural circuity, including the medulla oblongata, mid-brain, and cerebral cortex, within the production of thirst is a more likely mechanism \cite{mckinley_water_2006}. Since this original idea, further refinement into cortical sites active in thirst response have been elucidated. In both animal \cite{robinson_alimentary_1968} and human work \cite{egan_neural_2003}, common structures have appeared such as the anterior cingulate cortex. 


\section{Methods}

\subsection{MRI Scanning and Cognitive Motor Task}
Subjects completed the motor pacing task (MPT, described below) in between the T1 and T2 scan. During the MPT, blood oxygen level dependent (BOLD) responses were measured using an echo-planar imaging sequence with a total of 714 volumes (TR = 2000 ms, TE = 30 ms, flip angle = 90o, field of view = 204 x 204 mm${^2}$, in-plane resolution of 3 x 3 mm${^2}$). During the functional MRI (fMRI) scanning, subjects completed the MPT (\textit{E*Prime, Psychology Software Tools, Sharpsburg, PA}) requiring visually-paced rhythmic finger tapping with the right index finger. In the scanner, subjects lay supine and had viewed a display monitor (\textit{Silent Vision 6011, Avotec, Stuart, FL}) via a mirror placed on the head coil. Headphones (\textit{Silent Scan 3100, Avotec, Stuart, FL}) were placed on the subject and adequate visibility of the monitor was confirmed before each scan. If required, vision was corrected using MRI-compatible lenses. Some subjects (n = 3) did not undergo MRI scanning, and instead completed the MPT in a MRI simulator (\textit{Psychology Software Tools, Sharpsberg, PA}) built to mimic conditions within the MRI scanner.

The MPT (Figure \ref{fig:mpt}) consisted of 1 Hz alternating stimuli (yellow square presented for 500 ms) and fixation crosses with two pacing variations: i) regularly paced (MPTr; fixation cross for 500 ms) and ii) irregularly paced (MPTi; fixation cross presented for 400-600ms). Subjects were instructed to respond to the stimulus (yellow square) by pressing a button box (\textit{FORP 4 Diamond, Current Designs, Philadelphia, PA}). Errors were encoded binomially: ‘0’ (missed response) or ‘1’ (correct response). Blocks of thirty stimuli (all either MPTr or MPTi) were followed by 30 s rest. Twenty total blocks were completed (n = 600 stimuli) with extended (120 s) rest periods every five blocks, with total test duration equaling ~ 22 min. Block presentation (MPTr or MPTi) was randomized for each test iteration. Two behavioral measures were examined: accuracy (percentage correct responses) and reaction time (stimulus presentation to button press). The nature of the MPT indicated that only correct responses were examined for reaction time. Both reaction time and accuracy were averaged across each 5 min block of tie (quartiles).

\begin{figure}
	\centering
	\includegraphics[width=4in]{figures/"MPT".pdf}
	\caption{Motor Pacing Task paradigm. A) Regular task variation (500 ms inter-stimulus interval); B) irregular task variation (400-600 ms inter-stimulus interval with pseudo-average of 500 ms)}
	\label{fig:mpt}	
\end{figure}

\subsection{BOLD Analysis}
fMRI data analysis was completed using FSL (\textit{www.fmrib.ox.ac.uk/fsl}) \cite{smith_advances_2004}. All data were preprocessed by motion correcting images (MCFLIRT) \cite{jenkinson_improved_2002}, removing non-brain tissue (BET) \cite{smith_fast_2002}, distortion-corrected with a fMRI field map using PRELUDE and FUGE \cite{smith_advances_2004}, spatially smoothed using a Gaussian kernel of 8 mm full-width half maximum, and high pass temporal filtering (sigma = 100 s). First level (time-series) analysis was completed with a generalized linear model (FILM) including nonparametric estimation of time series autocorrelation \cite{woolrich_temporal_2001}. fMRI data for each subject were analyzed in native space (i.e., individual subject brain) before being initially registered to a high resolution structural image and then a nonlinear registration to standard MNI space (Montreal Neurological Institute; Montreal, Quebec, Canada) using FNIRT. All blood oxygen level dependent (BOLD) signals were measured as signal intensity compared to the rest periods. Time series analysis was completed with the contrasts of the entire task (MPTr and MPTi) along with further analysis isolating MPTr and MPTi.

Higher level analyses (i.e., across subjects and sessions) was completed using mixed effects (FLAME 1+2) which uses Markov Chain Monte Carlo sampling to identify true random-effect variance and degrees of freedom for each voxel. The main analysis examined the BOLD responses of the MPT, MPTr, and MPTi during each session. To compare across sessions, Z statistic images were produced by applying a cluster threshold of Z > 2.3 and (corrected) cluster significance threshold of p = 0.05 \cite{worsley_three-dimensional_1992}. Statistical maps were overlaid onto a standard brain template using MRIcron \cite{rorden_improving_2007} with a threshold of Z = 2.3.

\subsection{Statistical Analysis}
MPT accuracy and reaction time were analyzed with a linear mixed effects model with fixed effects of trial, task version (MPTr, MPTi), and time quartile. If a significant main or interaction effect was observed, post-hoc contrasts using Bonferroni-Holm corrections were calculated using the lsmeans package in R (https://cran.r-project.org/web/packages/lsmeans). The alpha level was set a priori as p < 0.05 to indicate statistical significance. Data are presented as mean ± SD.

\section{Results}
All hydration biomarkers, physiological, and perceptual responses for all of the trials can be found in Chapter 2: Results. The results below will report results specific to Chapter 3, namely the functional and performance results during the MPT.

\subsubsection{Cognitive-Motor Performance}
No significant session effect was observed for reaction time (CON: 151.7 ± 48.1, EHS: 149.1 ± 54.4, EHS-HYPO: 150.2 ${\pm}$ 50.7 ms; p = 0.81). Reaction time was faster during the fourth quartile (last 5 min block) compared to the first quartile (by 26.2 ${\pm}$ 17.7 ms, p ${<}$ 0.001) and also for the regularly-paced intervals (MPTr) of the test compared to MPTi (by 15.5 ${\pm}$ 9.3 ms, p ${<}$ 0.0001).    

Figure \ref{fig:mpt_accuracy_time} presents the MPT accuracy during each 5-min time block (quartile) for all trials. For the entire MPT, EHS-HYPO (67.3 ${\pm}$ 25.3) elicited lower accuracy compared to CON (81.1 ${\pm}$ 17.3\%; p = 0.004, ES: 0.74) but was not different vs. EHS (75.2 ${\pm}$ 21.4\%; p = 0.17, ES: 0.42). Overall, accuracy during EHS and CON was not different (p = 0.17, ES: 0.44). A significant main effect for time was observed, with lower accuracy occurring over the last compared to the first quartile (by 9.4 ${\pm}$ 6.8\%, p = 0.0001, ES: 0.83) although no time by trial interaction was observed (p = 0.74). Figure \ref{fig:mpt_individual} presents the individual responses during the MPT. Accuracy for MPTr (75.9 ${\pm}$ 23.5\%) and MPTi (72.8 ${\pm}$ 21.0\%) were not significantly different (p = 0.08, ES = 0.28). Accuracy by pacing variation with EHS-HYPO (67.5 ${\pm}$ 26.6, 67.1 ${\pm}$ 23.3\%), EHS (77.9 ${\pm}$ 23.6, 72.4 ${\pm}$ 18.9\%), and CON (82.8 ${\pm}$ 18.3, 79.3 ${\pm}$ 16.1\%) did not differ during MPTr and MPTi, respectively (p = 0.18).  

\begin{figure}
	\centering
	\includegraphics[width=6.3in]{figures/"mpt_accuracy_time".pdf}
	\caption{Mean ± SD accuracy (\%) throughout the 20 min motor pacing task during resting control (CON), exercise heat stress with fluid replacement (EHS), and exercise heat stress coupled with hypohydration (EHS-HYPO). Pound: p ${<}$ 0.05 vs CON}
	\label{fig:mpt_accuracy_time}	
\end{figure}

\begin{figure}
	\centering
	\includegraphics[width=7in]{figures/"mpt_individual_responses".pdf}
	\caption{Mean (\%) individual responses for the entire 20 min motor pacing task during resting control (CON), exercise heat stress with fluid replacement (EHS), and exercise heat stress coupled with hypohydration (EHS-HYPO). Line color indicates subjects where EHS-HYPO accuracy was either impaired (black) or equivalent/improved (grey) compared to CON.}
	\label{fig:mpt_individual}	
\end{figure}

\subsubsection{Brain Function During MPT}
BOLD responses during CON are presented to illustrate task-dependent neural resource requirements during the MPT (Figure \ref{fig:mpt_baseline_bold}). Elevated BOLD responses were observed for the following brain areas: bilateral sensorimotor, bilateral visual cortices, supplementary motor, left motor cortex, and bilateral basal ganglia. 

\begin{figure}
	\centering
	\includegraphics[width=6.3in]{figures/"mpt_baseline_bold".pdf}
	\caption{Axial slices of significantly (Z ${\ge}$ 1.6 with cluster correction of p ${<}$ 0.05) elevated blood oxygen level dependent (BOLD) responses throughout the motor pacing task (MPTr and MPTi) during the resting control (CON) trial. Color gradient indicates level of elevated BOLD responses.}
	\label{fig:mpt_baseline_bold}	
\end{figure}

BOLD response contrasts for the entire MPT are presented in Table \ref{tbl:mpt_entire_tbl} and Figure \ref{fig:mpt_entire_task_bold}. No differences in BOLD responses were observed between EHS and CON. EHS-DEH elicited two elevated BOLD clusters (p ${<}$ 0.05) compared to CON. The first cluster was located within right cerebral white matter and extended into the left cerebral white matter, the bilateral thalamus, and right subcortical grey matter structures while the second cluster was located within the left inferior temporal lobe and extended further into the left temporal lobe, bilateral cerebral white matter, and left subcortical grey matter structures (Figure \ref{fig:mpt_entire_task_bold}). EHS-HYPO also had elevated BOLD responses (p ${<}$ 0.05) vs. EHS within the left middle temporal lobe extending into the left parietal lobe, left occipital lobe, left cerebral white matter, and left subcortical grey matter structures (Table \ref{tbl:mpt_entire_tbl}, Figure \ref{fig:mpt_entire_task_bold}).


\begin{figure}
	\centering
	\includegraphics[width=6.3in]{figures/"mpt_entire_task_bold".pdf}
	\caption{Significantly elevated (Z ${\ge}$ 2.3 with cluster correction of p ${<}$ 0.05) blood oxygen level dependent (BOLD) responses for the entire motor pacing task during exercise heat stress with hypohydration (EHS-HYPO) compared to resting control (CON) and exercise heat stress without hypohydration (EHS). Areas of color indicate locations where EHS-HYPO elicited greater BOLD responses compared to CON (yellow) or EHS (purple). No differences were observed between CON and EHS (p ${>}$ 0.05)}
	\label{fig:mpt_entire_task_bold}	
\end{figure}

\begin{table}
	\caption{Significantly elevated (Z ${\ge}$ 2.3) BOLD responses observed throughout the entire motor pacing task for exercise heat stress with hypohydration (EHS-HYPO) compared to resting control (CON) and exercise heat stress without hypohydration (EHS). Cluster peaks are presented in MNI152 coordinates. Italicized areas indicate task-specific areas as identified during CON.}
	\centering
	\begin{tabular}{lll} 
		\hline
		& Regions in Cluster & Hemisphere \\
		\hline
		\textbf{EHS vs. CON} & No significant clusters & \\
		&& \\
		\textbf{EHS-HYPO ${>}$ EHS} && \\
		\underline{Cluster 1 Peak: 10, -34, 16} & \textit{Cerebral White Matter} & \textit{Right}  \\
		Voxels: 1954 & \textit{Cerebral White Matter} & \textit{Left} \\
		Peak Z: 3.81 & Hippocampus & Right \\
		& \textit{Pallidum} & \textit{Right} \\
		&\textit{Thalamus} & \textit{Right} \\
		& \textit{Thalamus} & \textit{Left} \\
		&& \\
		\underline{	Cluster 2 Peak: -46, -18, -20} & Amygdala & Left \\
		Voxels: 3185 & \textit{Cerebral White Matter} & \textit{Right} \\
		Peak Z: 3.34 & \textit{Cerebral White Matter} & \textit{Left} \\
		& Frontal Lobe, Pole & Right \\
		& Hippocampus & Left \\
		& Insular Cortex & Left \\
		& \textit{Putamen} & \textit{Left} \\
		& Temporal Lobe, Inferior Gyrus & Left \\
		& Temporal Lobe, Middle Gyrus & Left \\
		& Temporal Lobe, Pole & Left \\
		&& \\
		\textbf{EHS-HYPO ${>}$ CON} & & \\	
		\underline{Cluster Peak: -46, -4, -2} & \textit{Cerebral White Matter} & \textit{Left} \\
		Voxels: 2382 & Insular Cortex & Left \\
		Peak Z: 3.75 & \textit{Occipital Lobe, Lateral Division} & \textit{Left} \\
		& Parietal Lobe, Operculum Cortex & Left \\
		& Parietal Lobe, Angular Gyrus & Left \\ 
		& Parietal Lobe, Supramarginal Gyrus & Left \\
		& \textit{Putamen} & \textit{Left} \\
		& Temporal Lobe, Middle Gyrus & Left \\
		& Temporal Lobe, Planum Polare & Left \\
		& \textit{Thalamus} & \textit{Left} \\
		\hline		    
	\end{tabular}
	\label{tbl:mpt_entire_tbl}
\end{table}

Further analyses of the BOLD responses were made based to separate the different timing interval variations. No differences in MPTi BOLD responses were observed among trials. During the MPTr, no differences in BOLD responses were observed between EHS and CON. However, EHS-HYPO elicited elevated BOLD responses vs. CON (p ${<}$ 0.05) during MPTr within the right thalamus/basal ganglia, hippocampus, amygdala, and brain stem (Table \ref{tbl:mpt_reg_tbl}, Figure \ref{fig:mpt_regular_bold}). In addition, EHS-HYPO elevated BOLD responses vs. EHS (p ${<}$ 0.05) within the right cerebral white matter, caudate, and putamen along with the frontal lobe and anterior cingulate gyrus (Table \ref{tbl:mpt_reg_tbl}, Figure \ref{fig:mpt_regular_bold}). 

\begin{figure}
	\centering
	\includegraphics[width=6.3in]{figures/"mpt_regular_bold".pdf}
	\caption{Significantly elevated (Z ${\ge}$ 2.7 with cluster correction of p ${<}$ 0.05) blood oxygen level dependent (BOLD) responses during regularly paced stimuli within the motor pacing task (MPTr) for exercise heat stress with hypohydration (EHS-HYPO) compared to resting control (CON) and exercise heat stress without hypohydration (EHS). Areas of color indicate locations where EHS-HYPO elicited greater BOLD responses compared to CON (yellow) or EHS (purple)}
	\label{fig:mpt_regular_bold}	
\end{figure}

\begin{table}
	\caption{Significantly elevated (Z ${\ge}$ 2.7) BOLD responses observed during the regularly paced stimuli within the motor pacing task (MPTr) for exercise heat stress with hypohydration (EHS-HYPO) compared to resting control (CON) and exercise heat stress without hypohydration (EHS). Cluster peaks are presented in MNI152 coordinates. Italicized areas indicate task-specific areas as identified during CON.}
	\centering
	\begin{tabular}{lll} 
		\hline
		& Regions in Cluster & Hemisphere \\
		\hline
		\textbf{EHS vs. CON} & No significant clusters & \\
		&& \\
		\textbf{EHS-HYPO ${>}$ CON} && \\
		\underline{Cluster Peak: 16, -20, 12} & Amygdala & Right  \\
		Voxels: 1512 & Brain Stem & \\
		Peak Z: 4.62 & \textit{Caudate} & \textit{Right} \\
		& \textit{Cerebral White Matter} & \textit{Right} \\
		& Cingulate Gyrus, Posterior & Right \\
		& Hippocampus & Right \\
		& Insular Cortex & Right \\
		& \textit{Pallidum} & \textit{Right} \\
		& Parahippocampal Gyrus & Right \\
		& Parietal Lobe, Operculum Cortex \\
		& \textit{Thalamus} & \textit{Right} \\
		&& \\
		\textbf{EHS-HYPO ${>}$ EHS} & & \\	
		\underline{Cluster Peak: 24, 10, 26} & Anterior Cingulate Gyrus & \\
		Voxels: 1131 & \textit{Caudate} & \textit{Left} \\
		Peak Z: 4.90 & \textit{Cerebral White Matter} & \textit{Left} \\
		& Frontal Lobe, Inferior Frontal Gyrus & Right \\
		& Frontal Lobe, Superior Frontal Gyrus & Right \\ 
		& Frontal Pole & \\
		& Insular Cortex & Right \\
		& Supplementary Motor Cortex & \\
		& \textit{Putamen} & \textit{Right} \\
		\hline		    
	\end{tabular}
	\label{tbl:mpt_reg_tbl}
\end{table}

\section{Discussion}

One aspect of the MPT which is may influence results is the sustained attention requirements. Sustained attention, the requirement for attentional control towards a stimulus, is more difficult in monotonous versus complex situations \cite{langner_sustaining_2013}. However, many view sustained attention as requiring ${\ge}$ 10s of stimulus processing (i.e., stimulus detection window) \cite{langner_sustaining_2013}, which is not present with the MPT. However, sustained attention appears to involve right fronto-parietal processing network \cite{robertson_vigilant_2010}, which, was observed under resting control conditions. However, a right-lateralized network including the thalamus, anterior cingulate cortex, and other fronto-parietal areas has been identified as being intrinsically active in sustaining attention \cite{sturm_functional_2001}. This has implications for the current study, as it integrates with subcoritcal regions to maintain attention to meet task demands \cite{robertson_vigilant_2010}. Therefore, it could be that the bilateral BOLD responses could also be elevated effort required to elicit the requisite levels of attention. However, this does not appear to affect performance as there was no exaggerated performance degradation over time. Therefore, we reject the initial hypothesis that HYPO would exacerbate time-related performance. Furthermore, this only appears to take 30 min of performance \cite{watson_mild_2015}.       



