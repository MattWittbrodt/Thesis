\chapter{Aim 3}

\section{Abstract}

\section{Introduction}

\section{Methods}

\subsection{Participants}
[SEVEN] right-handed healthy males (age: 24.5 ± 4.7 y, body mass: 62.5 ± 6.3 kg, body fat: 14.5 ± 2.9\%) participated in the study. All subjects engaged in regular exercise (${\geq}$ 4 d/wk). All subjects served as their own control and completed each condition. All subjects gave written, informed consent as approved by the Georgia Institute of Technology Institutional Review Board.

\subsection{Experimental Design}
Subjects completed three preliminary sessions and three experimental trials, all within ${\sim}$3 weeks. Subjects were tested in Atlanta, GA, not within the winter months (mid-December to mid-March). Before all sessions, subjects were instructed to consume liberal ($>$ 500 mL) and consistent fluid the night before, abstain from alcohol for the previous 12 h, and enter the laboratory after an overnight fast. Three preliminary sessions were conducted to establish baseline body mass (BM), plasma osmolality (POsm), and urine specific gravity (USG; \textit{ATAGO USA, Bellevue, WA}) as previously recommended \cite{cheuvront_biological_2010}. During one preliminary session, a graded exercise test was completed by subjects to established aerobic exercise capacity and estimate sweat rate. Before each graded exercise test, the metabolic cart (\textit{Parvomedics, Sandy, UT}) was calibrated using a 3-L volume syringe and gases of known concentrations (16\% O2, 4\% CO2). The graded exercise test (Figure \ref{fig:vo2_max}) was a modified Bruce protocol \cite{american_college_of_sports_medicine_acsms_2013} consisted of selecting a brisk walking pace (${\sim}$3.5 mph) and two minute stages of 2.5\% increasing grade until 7.5\%. The gradient was then clamped at 7.5\% and running speed increased 1mph every two minutes until volitional fatigue (i.e., cessation by participant). All participants were required to achieve $>$60th percentile of aerobic fitness (45 ml/kg/min) \cite{american_college_of_sports_medicine_acsms_2013}.    

Following the preliminary sessions, subjects completed three experimental trials: control (CON; no exercise-heat stress), exercise-heat stress with fluid replacement (EHS), and exercise-heat stress with hypohydration (EHS-HYPO; exercise-heat stress without fluid replacement). The order of experimental trials was counterbalanced. The experimental trials were initiated in the morning (~0700) and first morning BM, USG, UOsm, and POsm were assessed to ensure adequate hydration status (${\leq}$1\% difference in BM from preceding 3 d average) \cite{cheuvront_biological_2010}. Subjects then consumed a nutrition bar (250 kcal) and water (150 ml) about 20 min before entering the hot (EHS, EHS-HYPO; 45${^o}$C, 15\% RH) or temperate (CON; 22${^o}$C, 30\% RH) environments. For EHS and EHS-HYPO, the exercise consisted of walk (45 min): rest (15 min) cycles for 150 min at a workload eliciting an initial heart rate of ${\sim}$110-120 bt/min (${\sim}$3.5 mph, 5\% grade). The EHS and EHS-HYPO trials were matched for exercise duration and intensity. The goal during EHS-HYPO was to achieve ${\sim}$3\% BM loss. During EHS, subjects consumed a fluid volume equivalent to sweat loss, while no water was consumed during EHS-HYPO. Following EHS and EHS-HYPO, subjects were removed to the temperate environment to cool for ${\sim}$30 min prior to a final BM, blood glucose, and POsm. Subjects were then moved to the EEG facility (${\sim}$ 1 minute walk) with a total recovery period of 45 min. During CON, subjects reported under same baseline conditions with meal provided but instead of walking in the heat, sat quietly for ${\sim}$1.5 h in the temperate environment while abstaining from mentally stimulating activities before being transported to the MRI.

\subsection{Physiological and Perceptual Measures}
During all exercise trials, heart rate (HR) and rectal temperature (YSI, Yellow Springs, OH) were measured at 5 min intervals, and did not exceed 90\% of age-predicted HRmax (220-age) or 39.5${^o}$C, respectively.  Blood samples were obtained by finger puncture on a heated digit after being seated for 10 min. POsm was determined from the median of at least three measurements (median of five if variation exceeded 1\%) using freeze point depression (\textit{Osmette II, Precision Systems, Natick, MA}) as described previously \cite{wittbrodt_biological_2015}. Blood glucose was measured (\textit{OneTouch UltraMini, LifeScan Inc., Wayne, PA}) post-exercise (${\sim}$3 h after the meal) for EHS-HYPO and EHS and ${\sim}$90 min after the meal for CON. Nude, dry BM was measured before and after each hour period of exercise on a digital platform scale and appropriately corrected for urine output. During EHS-HYPO, subjects were blinded to their BM. Rating of Perceived exertion (RPE, Figure \ref{fig:rpe}) \cite{borg_psychophysical_1982}, thirst (1-10 Likert scale, Figure \ref{fig:thirst}), and core affect (\textit{Feeling Scale, Felt Arousal Scale}) \cite{hardy_not_1989,gauvin_exercise-induced_1993} were also assessed at five min intervals.

\subsection{EEG and Visuomotor Task}
Subjects were placed in the 3T MRI (Siemens Trio, Siemens, Germany) scanner with the 12-channel head coil affixed and head position in a way to minimize movement in the X, Y, or Z axes. The scanning sequence consisted of a T1-MPRAGE with 256 slices and 1.0 x 1.0 x 1.0mm voxel size (TA: 6.17s, 9o flip angle, TI: 850ms, TR: 2250ms; TE: 3.98ms) and a T2 Space with 1.0 x 1.0 x 1.0mm voxel size (TA: 4.43s, TR: 3200ms, TE: 428ms). In between the T1 and T2 scan, subjects completed the motor pacing task (described below) during which blood oxygen level dependent (BOLD) responses were measured using an echo-planar imaging sequence with a total of 714 volumes (TR = 2000 ms, TE = 30 ms, flip angle = 90o, field of view = 204 x 204 mm2, in-plane resolution of 3 x 3 mm2). 

During the functional MRI (fMRI) scanning, subjects completed a visuomotor pacing task (VMPT; E*Prime, Psychology Software Tools, Sharpsburg, PA) requiring visually-paced rhythmic finger tapping with the right index finger. In the scanner, subjects lay supine and viewed a display monitor (Silent Vision 6011, Avotec, Stuart, FL) via a mirror placed on the head coil. Headphones (Silent Scan 3100, Avotec, Stuart, FL) were placed on the subject and adequate visibility of the monitor was confirmed before each scan. If required, vision was corrected using MRI-compatible lenses. Some subjects (n = 3) did not undergo MRI scanning, and instead completed the VMPT in a MRI simulator (Psychology Software Tools, Sharpsberg, PA) built to mimic conditions within the MRI scanner.

The PCRT, based on previous research \cite{stetson_early_2015,nobuyuki_timing_2003} and similar to serial choice reaction tasks \cite{praamstra_neurophysiology_2006}, will require subjects to observe a stimulus in the center of the screen (Figure \ref{fig:pcrt}). Subjects were instructed to press the right button if a red stimulus (circle) appears and the left button if a green stimulus is displayed. This task will incorporate two important features that distinguish it from Study 1: i) responses in the form of bimanual reaching and ii) unpredictable stimuli (weighted and non-uniform stimulus display). Subjects will be presented with one of two stimuli (reach right or left) in a rhythmic fashion, creating a decision process requiring evidence accumulation by the subject (motor planning) before enacting the motor program of reaching either left or right \cite{wong_motor_2015}. Each session will have a randomly selected `dominant' direction (right or left), and the dominance weight will be variable from block to block: 50\%, 66\%, 84\%, and 100\%. Each of the four blocks (dominance percentage) will include 300 stimuli, for a total of 1200 stimuli per session. Compared to the MPT, the PCRT will require greater activation of the motor planning system, resulting in elevated activity within the parietal sensory association cortex, premotor cortex, supplementary motor area, basal ganglia, and lateral cerebellum \cite{wing_response_1973}. The task will be created using the PsychoPy application for the Python Programming Language (\textit{www.psychopy.org}). 

\begin{figure}
	\centering
	\includegraphics[width=2.5in]{instruments/"pcrt".pdf}
	\caption{Schematic of Probabilistic Choice Reaction Task}
	\label{fig:pcrt}
\end{figure} 


\subsection{EEG Acquisition and Processing}
EEG measures electrical activity from the brain via electrodes placed on the scalp. As cortical regions are active, extracellular current creates inhibitory and excitatory post synaptic potentials which are detected by the scalp electrodes \cite{kuperberg_electroencephalography_2004}. Subjects will be fitted with a standard 58-channel EEG cap (\textit{Electrocap, Eaton, OH}) with a sampling rate of 1000 Hz \textit{Synamps 2, Neuroscan, Charlotte, NC, USA}). Brain activity will be measured during the performance of the Probabilistic Choice Reaction Task (PCRT, described below) by responding to images on the screen on a custom built, two button keypad (\textit{Arduino Uno, www.arduino.cc}). Eye blinks and movement artifacts will be recorded using electrooculographic (EOG) activity and subtracted from the data offline. Data will initially be analyzed by examining changes in the event-related potentials (ERP) compared across each condition but also over time. This allows for investigation into specific sensory and cognitive responses to on-screen stimuli and movement generation.  

The raw, continuous EEG was leaded into EEG lab \cite{delorme_eeglab:_2004} for subsequent post-processing and analysis. All data was first high-pass filtered at 0.5 Hz and low-pass filtered at 50 Hz. Bad channels (those with maximum), bad channels were identified, rejected and interpolated. Data was then epoched from 100 ms pre- stimulus onset to 500 ms post-stimulus for all 945 stimuli presentations with baseline correction from 100 to 0 msec, to provide epochs for subsequent event-related potential (ERP) analysis. An independent component analysis (ICA) was then conducted, utilizing EEGLAB’s runica algorithm, to assist in removal of blink and other stereotypical movement artifact components. Selection of components for removal was based on visual inspection of scalp map localization, unusual spectral frequency patterns and irregular ERP-image activity. An average of 7 of 58 components per participant was selected for subtraction. After removal of artifact components, data was segmented into 27 separate datasets, one for each block. A separate dipole fitting analysis was then conducted, utilizing EEGLAB’s DIPFIT plugin [21], for every block for each subject. Dipole localization was determined utilizing the Talairach Client application [22]. After pre-processing, all datasets were loaded into a STUDY structure for group analysis.

\subsection{Statistical Analysis}
All PCRT data were analyzed using a linear mixed effects model with lmer() in R (\textit{https://cran.r-project.org/web/packages/lme4/}).



\section{Results}

\textit{PCRT}








\section{Discussion}

\section{Conclusion}