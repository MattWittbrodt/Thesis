\chapter{Summary and Future Directions}

\section{Integration of Doctoral Thesis Findings}


\section{Future Directions}

Although this Doctoral Thesis has presented novel findings regarding the influence of hypohydration on cognitive performance, vast amounts of future research possibilities exist. 

\subsection{Elderly Populations}

While the current thesis investigated adults (age: 18 - 45), further investigations should take place within populations susceptible to daily HYPO should occur. Elderly are at-risk for HYPO and have a high prevalence of confusion \cite{mentes_getting_1997}. 

Furthermore, there appears to be a relationship within a clinical population (multiple sclerosis) which suggests greater levels of fatigue possibly translating into medical affects(Cincotta 2016).


\subsection{Potential Predictors of Cognitive Impairment}

It is well understood that hypohydration elicits an absolute hypovolemia, which . Physiological measures, such as decreased cerebral blood flow velocity \cite{carter_hypohydration_2006}.

\subsection{Timing and Rehydration Strategies}

It is clear that different rehydration strategies vary in effectiveness given the volume, duration, and composition of said strategy  EVANS REVIEW. 

\subsection{Multifactoral Stress}

Multifactoral stress can cause additive and severe cognitive deficits \cite{lieberman_severe_2005}.